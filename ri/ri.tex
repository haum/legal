\documentclass[a4paper, 11pt]{article}

    % Document largement inspiré du règlement intérieur du LOL :
    %     http://labolyon.fr
    %
    % Merci à eux !
    %
    % Versions :
    % [0.1] HAUM
    %     sam. oct. 13 15:49:07 CEST 2012

    \usepackage[utf8]{inputenc}
    \usepackage[T1]{fontenc}
    \usepackage{geometry}
    \geometry{top=3cm, bottom=2cm, left=2cm, right=2cm}

    \title{Règlement Intérieur}
    \author{}
    \date{}

    \renewcommand{\thesection}{Article \arabic{section}~~~-}
    \newcommand{\nomHS}{HAUM}
    \newcommand{\dateAGC}{mardi 13 novembre 2012}
    \newcommand\sep{\noindent\rule{\linewidth}{.5pt}}

\begin{document}

    \maketitle

\section{Objet} % {{{

Le présent règlement intérieur complète et précise les statuts de l'association «\nomHS», approuvés par l'assemblée
consitutive du \dateAGC.

% }}}

\section{Adhésion} % {{{

L'adhésion d'un mineur ne peut avoir lieu que sous la responsabilité et avec l'accord de son ou ses tuteurs légaux.

Le futur membre actif doit présenter sa demande d'adhésion au bureau par un moyen à sa convenance.
Celle-ci est évaluée lors de la réunion du bureau.

% }}}

\section{Durée d'adhésion et modalités de cotisation} % {{{

% ça simplifie le boulot du trésorier : 
L'adhesion est valable de la date à laquelle elle est contractée jusqu'au prochain 31 décembre.

% }}}

\section{Procédure de radiation} % {{{

L'appellation \textit{"motif grave"}  comprend notament le non-paiement d'une cotisation à sa date d'exigibilité et la mise en danger d'autrui par non-respect de l'article \textbf{Sécurité} du présent règlement.

Lors de la procédure de radiation pour motif grave, le ou les membres responsables doivent pouvoir s'exprimer devant le
bureau avant que celui-ci décide ou non de prononcer la radiation.

% }}}

\section{Réunion du bureau} % {{{

Le bureau se réunit au moins une fois par année calendaire sur convocation du président ou du trésorier, aussi souvent
que nécessaire sur la demande de la moitié de ses membres, de la moitié des membres actifs, ou en cas de besoin de
statuer sur l'adhésion d'un membre.

La convocation doit être transmise par courrier électronique ou postal aux membres du bureau au moins une semaine avant
la date de réunion ainsi qu'à l'ensemble des membres.

Tout membre du bureau absent sans excuse à 3 réunions consécutives pourra être considéré comme démissionnaire.
Les membres du bureau ne peuvent être représentés par procuration.

Le bureau peut se réunir physiquement ou par n'importe quel moyen approuvé par la totalité de ses membres.

Le quorum est fixé aux deux tiers des membres du bureau.
Les réunions font l'objet d'un procès-verbal.

% }}}

\section{Procuration} % {{{

Un membre actif indisponible à la date d'une assemblée générale peut se faire représenter par un autre membre actif
auquel il aura confié une procuration.

La procuration doit clairement identifier le mandant et le mandataire.

Un membre actif ne peut être mandataire de plus de deux procurations.

% }}}

\section{Vote} % {{{

Lors de l'élection du bureau en assemblée générale, les candidats sont approuvés individuellement \textit{via} un vote à
la majorité simple des membres actifs présents ou représentés.
L'assemblée générale ne statue pas sur la répartition des rôles dans le bureau.

Lors d'un vote au sein du bureau, le vote se fait à la majorité simple des membres du bureau.
Si une égalité survient, la voix du président est prépondérante.

% }}}

\section{Sécurité} % {{{

L'accès et l'utilisation des ressources de l'association sont soumis à l'acceptation des règles de sécurité ci-dessous.

Les membres sont responsables de leur propre sécurité et doivent à ce titre :

\begin{itemize}
    \item s'informer des risques associés à la mise en oeuvre des équipements qu'ils souhaitent utiliser;
    \item s'équiper personnellement des EPI\footnote{Equipement de Protection Individuelle} recommandés lors de l'usage
        de ces équipements quand l'association n'est pas en mesure de les leur fournir;
    \item s'assurer que ces EPI sont aux normes en vigueur, en bon état et de taille adaptée.
\end{itemize}

Par ailleurs, un membre mettant à disposition des autres un équipement dont l'usage présente des risques a le devoir :

\begin{itemize}
    \item d'informer les autres membres des risques associés à la mise en oeuvre dudit équipement;
    \item de s'assurer de la maintenance de l'équipement et en particulier des dispositifs de sécurité intégrés;
    \item de marquer l'équipement de manière à signaler les risques (pictogrammes normalisés, affiches
        informatives\ldots)
\end{itemize}

L'association ne pourra être tenue pour responsable des problèmes liés à un non-respect des consignes de sécurité.

% }}}

\section{Facture des services et consommables aux personnes non-membres}

Toute personne extérieure à l'association voulant bénéficier des
équipements de celle ci devra s'acquitter de 5 EUR par heure entamée
d'exploitation.


La manipulation est réalisée systématiquement par un membre actif de
l'association.

\section{Don, legs, prêt et cession de biens} % {{{

Le don et le prêt de matériaux et de matériels sont autorisés et même encouragés. Ces matériaux et matériels représentent l'équipement du hackerspace.

\begin{itemize}
    \item Les donateurs doivent respecter la législation en vigueur vis à vis du transport et du stockage.
    \item Ils ne doivent pas encombrer inutilement le hackerspace;
    \item Ils ne doivent pas représenter un danger direct pour les utilisateurs du hackerspace.
\end{itemize}

Les membres du hackerspace se réservent le droit de demander le retrait ou d'éliminer tout objet qui ne respecterait pas ces règles.

% Ce qui suit est du commentaire éliminé automatiquement :
% En cas d'élimination de matos :
% celui ci ne *doit pas revenir à un membre de l'asso !*
% C'est puni par la loi !
L'élimination de matériel peut avoir lieu de deux manières :

\begin{itemize}
    \item Si le matériel est un don, alors celui-ci sera éliminé selon les normes en vigueur ou donné à une association poursuivant les mêmes buts ou à une oeuvre caritative;
    \item Si le matériel est un prêt alors le prêteur se verra demander le retrait du matériel.
\end{itemize}

Les utilisateurs s'engagent à ne pas emprunter l'équipement du hackerspace sans l'autorisation des responsables. Les prêts doivent rester exceptionnels et ne pas gêner le fonctionnement du hackerspace.

% }}}

\section{Révision du règlement intérieur} % {{{

Tout projet visant à modifier le règlement intérieur de l'association doit être soumis à l'approbation de l'assemblée
générale ordinaire ou extraordinaire.

% }}}
\bigskip\bigskip

\sep

\bigskip\bigskip

Le présent règlement a été approuvé par l'assemblée générale constitutive du \dateAGC.

\bigskip\bigskip

Président(e) :


\bigskip\bigskip

Trésorier(ère) :

\end{document}

