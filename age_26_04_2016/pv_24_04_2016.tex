\documentclass[a4paper, 11pt]{article}
	\usepackage[utf8]{inputenc}
    \usepackage[french]{babel}
    \usepackage[T1]{fontenc}
    \usepackage[top=3cm,, right=2cm,bottom=2cm,left=2cm]{geometry}
    
    \usepackage{tabularx}
    
    \title{Hackerspace Au Mans}
    \author{Procès-Verbal de l'Assemblée Générale Extraordinaire}
    \date{Le vendredi 26 Avril 2016}
    
    \newcommand\sep{\noindent\rule{\linewidth}{.5pt}}
    
    \newcommand{\vote}[5]{
    
    \smallskip
    \fbox{\begin{minipage}[l]{\textwidth}
    	\smallskip
        \begin{center}
        	\underline{\textsc{Vote}}
        \end{center}
        
        #1\\
        \textbf{Votants} #2\\
        \textbf{Pour} #3\\
        \textbf{Contre} #4\\
        \textbf{NSPP} #5
        
        \smallskip
        
    \end{minipage}}
        \medskip
    }
    
    \newcommand\question[1]{\underline{\textsc{#1}} : }
    
    %\underline{#2}}
    
    
\begin{document}

\maketitle

\section*{Préambule}

\section{Effectifs}

\begin{itemize}
	\item Membres présents : 
		\begin{itemize}
            \item CONTY Romuald
            \item FAVREAU Tifenn
            \item ROUSSEAU Fabien
            \item VALLEE Sébastien
            \item VANNIER Laurent 
            \item BREHERET Jérôme
            \item TOUCHARD Florent
            \item COSTE Jérôme
            \item QUATRAVAUX Fabien
            \item LEFRANÇOIS Jean-Marc
            \item PORTE Romain
            \item BRETON Corentin
            \item BATAILLE Fabien
            \item CHAINOT-BATAILLE Manuella
            \item GABORIT-LEBREQUER Matthieu
		\end{itemize}
	\item Absents ayant donné procuration à BATAILLE Fabien :
		\begin{itemize}
			\item GERVOT Suzy-Lou
		\end{itemize}
	\item Absents ayant donné procuration à VALLEE Sébastien :
		\begin{itemize}
			\item GABORIT Mathieu
			\item BLAIN Frédéric
		\end{itemize}
	\item Absents ayant donné procuration à PORTE Romain :
		\begin{itemize}
			\item BEN-ICHE Medhi
		\end{itemize}
	\item Non-membres présents :
    	\begin{itemize}
    	    \item FRESNEAU François
            \item TARTIERE Romain
    	\end{itemize}
\end{itemize}

\bigskip
\textbf{La séance est ouverte à 19h47}

L'assemblée décide par un scrutin à l'unanimité d'accorder le droit de 
vote aux non-membres présents. 

L'assemblée décide par un scrutin à l'unanimité moins une abstention de 
mandater l'actuel secrétaire pour la modification des statuts afin de 
refléter les modifications apportées au cours de l'AG.

\section{Changement de nom}

Le changement de nom est discuté de longue date et il avait été décidé 
de modifier le nom de manière officielle que lorsqu'un autre élément des 
statuts devrait être modifié également.

Le nom est, conformément au procès-verbal de l'Assemblée Générale 
Ordinaire de 2015, modifié de \textbf{HAUM (Hackerspace de l'Université 
du Maine)} en \textbf{HAUM}. 

\vote{Modification du nom de l'association}{14}{14}{0}{0}

\section{Changement de siège social}

Afin de refléter la réalité de l'action du HAUM au cours des dernières 
années, il est proposé de modifier le siège social pour le placer au 
\textbf{19 Boulevard Marie et Alexandre Oyon, 72100 Le Mans}.

\vote{Modification du siège social de l'association}{14}{14}{0}{0}

\section{Modification des options d'adhésion}

Pour permettre l'adhésion à moindre frais des sympathisants tout en 
continuant à encourager l'adhésion des membres souhaitant prendre 
activement part à la vie associative, il est proposé d'ajouter un 
nouveau palier d'adhésion.

En plus de l'actuel statut de membre actif, il est proposé un statut de 
\textbf{membre sympathisant} accessible par une cotisation de 10 EUR ou 
plus. Ce statut ne donne pas accès au droit de vote et le membre n'est 
par conséquent pas compté dans le quorum. Il ne donne pas accès aux 
tarifs préférentiels disponibles pour les membres actifs mais il permet 
d'aider le HAUM \textit{via} une augmentation de la base de membres et 
une contribution pécuniaire.

Afin de faciliter l'ajustement ultérieur des adhésions, il est décidé de 
placer la définition des membres et de la cotisation dans le règlement 
intérieur.

L'adhésion de personnes morales sous le statut de membre partenaire est 
désormais possible pour une adhésion minimale de 90 EUR. La possibilité 
d'action des membres partenaires au sein de l'association sera négociée 
au cas par cas entre l'entité adhérant et le bureau de l'association.

\vote{Modification des types d'adhésion et de leur emplacement de 
définition}{17}{17}{0}{0}

\section{Droit à l'image des membres}

Dans le cadre de la communication externe du HAUM, il peut être 
nécessaire de publier des images et des vidéos contenant des images de 
certains des membres de l'association. A ce titre, il est demandé au 
membre de voter pour l'inclusion au règlement intérieur d'un article 
autorisant le HAUM a utiliser l'image de ses membres dans sa 
communication externe.

\vote{Ajout d'un article concernant l'utilisation de l'image des membres 
dans le règlement intérieur}{17}{16}{1}{0}

\section{Modification du bureau}

Il est proposé de ne pas dissoudre l'actuel bureau de l'association mais 
de procéder à l'inclusion des volontaires et à une nouvelle répartition des rôles 
au sein de celui-ci.

Une réunion de bureau sera immédiatement convoquée suite à l'AG.

\vote{Inclusion de Tifenn Favreau au bureau}{17}{17}{0}{0}

\vote{Inclusion de Frédéric Blain au bureau}{17}{17}{0}{0}


La présente Assemblée Générale modifiant les statuts, ils seront 
redéposés en Préfecture une fois les modification inclues. 

La séance est levée à 21h45.

\end{document}

