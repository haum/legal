\documentclass[a4paper, 11pt]{article}

	\usepackage[utf8]{inputenc}
	\usepackage[T1]{fontenc}
	\usepackage[french]{babel}
	\usepackage{amsmath}

    \title{HAUM}
    \author{Procès Verbal de Réunion de Bureau}
	\date{18 Septembre 2014}


\begin{document}
	\maketitle

	\section{Ordre du Jour}

	\begin{itemize}
		\item Assemblée Générale Ordinaire
		\begin{itemize}
			\item Date
			\item Lieu
			\item Ordre du Jour
		\end{itemize}
		\item Communication autour des activités
		\item Reflexion autour du financement des projets
		\item Ouvertures du vendredi
	\end{itemize}

	\section{Présence}

	\subsection{Présents}

	MM. \textsc{Lechat} Thomas,
	\textsc{Breheret} Jérôme,
	\textsc{Gaborit} Mathieu.

	Mme \textsc{Dinsenmeyer} Alice.

	\subsection{Absents Excusés}

	MM. \textsc{Ereau} Baptiste,
	\textsc{Bourneuf} Lucas.

	\subsection{Spectateurs}

	Mme \textsc{Cudennec} Armelle.


	\section{Réunion}

	\bigskip

	Ouverture de la séance à 17h30.

	\bigskip

	\subsection{Assemblée Générale}

	Il est décidé de convoquer simultanément les assemblées générales Ordinaire et Extraordinaire pour pouvoir
	éventuellement répondre à un besoin de changement de statuts.

	L'Assemblée Générale se tiendra le 10 Octobre 2014 à la Ruche Numérique (19 Bd M\&A Oyon, 72000 Le Mans).

	La convocation partira au plus tôt et le dossier relatif à la tenue de l'assemblée générale suivra sous peu.

	L'ordre du jour est fixé comme suit :

	\begin{itemize}
			\item Rapport moral
			\item Objectifs pour l'association l'an prochain
			\item Rapport Financier
			\item Objectifs pour la gestion comptable l'an prochain
			\item Présentation et vote des amendements aux statuts et au RI
			\item Questions au bureau
			\item Election d'un nouveau bureau
			\item Questions Générales
	\end{itemize}

	La date limite de dépôt d'amendements est fixée au 4 Octobre à 12h.

	\subsection{Communication}

	Il est décidé de proposer un poste de "responsable communication". L'objectif pour le HAUM est de diffuser son
	actualité et ses informations de manière cohérente et efficace. Cela permettra de décharger certains membres de
	cette partie du travail tout en garantissant que quelqu'un le fasse.

	Il est prévu que le "Responsable communication" soit nommé au sein du bureau.
	Il sera notament en charge de l'alimentation des comptes Twitter et Flickr et de l'annnonce des évènements.

	\subsection{Financement des projets}

	Les moyens actuels de financement des projets ne sont pas efficaces et ne permettent pas la croissance du
	hackerspace. Il est décidé de former un groupe de réflexion autour du financement.

	Pour rappel, plusieurs solutions avaient déjà été envisagées : pots dédiés à chaque projet, financement sur un
	budget dédié, appel aux dons pour des projets particuliers, sposoring/partenariats.

	Après des projets comme le piano stairs et le pong 1D (qui sont des projets du HAUM lui même et non d'un de ses
	membres), on remarque que le financement par "pot" dédié au projet conduit à ce que certains participent énormément
	alors que d'autres pas du tout : cette solution n'est peut être pas la bonne.

	La réflexion devra être engagée avec la trésorerie.

	\section{Ouverture du vendredi}

	A. \textsc{Dinsenmeyer} et T. \textsc{Lechat} se sont proposés (sur la mailing-list) pour assurer l'ouverture du
	local le vendredi soir. La présidence a contacté la Ruche Numérique pour avoir plus de détail quand la la
	transmission des accès.

	En attendant l'élection d'un nouveau bureau (qui aura, par convention, les droits d'accès au bâtiment), les
	ouvertures du vendredi se feront avec la signature d'une convention ponctuelle.


	\bigskip

	Fin de la séance à 18h45.

	\bigskip


\end{document}

