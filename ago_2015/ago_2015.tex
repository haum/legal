\documentclass[11pt]{article}
\usepackage[french]{babel}
\usepackage[T1]{fontenc}
\usepackage[utf8]{inputenc}
\usepackage{url}

\usepackage{ulem} % to use a strikeout/strikethrough font
\usepackage{color}
\newcommand{\fs}[1]{\textcolor{red}{\sout{#1}}}
\newcommand{\f}[1]{\textcolor{blue}{#1}}

\usepackage[top=3cm,right=2cm,bottom=2cm,left=2cm]{geometry}

\title{HAUM}
\author{Assemblée Générale Ordinaire}
\date{10 Octobre 2014}

\begin{document}
\maketitle


\section*{Convocation}

Madame, Monsieur, 

L'association HAUM (loi 1901) vous convoque à ses Assemblées Générales Ordinaire et Extraordinaire qui se tiendront le :

\begin{center}
{\Large XXX Novembre 2015 à XXhXX}\\
à la Ruche Numérique, 19 Bd M\&A Oyon (Gare Sud), 72 000 Le Mans
\end{center}

En cas d'impossibilité, veuillez prévenir et veiller à vous faire représenter si vous le souhaitez (2 procurations maximum par personne).

\section*{Déroulement}

\begin{enumerate}
    \item Présentation de l'association
    \item Rapport moral
        \begin{enumerate}
            \item Bilan
            \item Objectifs
            \item Questions
        \end{enumerate}
    \item Rapport financier
        \begin{enumerate}
            \item Bilan
            \item Objectifs
            \item Questions
        \end{enumerate}
    \item Motions et vote
    \item Election du nouveau bureau
    \item Questions diverses
\end{enumerate}

\section{Présentation de l'association}
\section{Rapport Moral}


\subsection{Bilan}

On a coutume de dire que la première année est un test, la deuxième l'épreuve réelle et la troisième le début de l'age
d'or.... Si le HAUM continue sur sa lancée, cet adage pourrait se vérifier.

Cette troisième année d'exercice a été éprouvante pour l'association mais cela semble porter ses fruits et les résultats
commencent à apparaître.

En 2015, le HAUM est passé de XX à plus de 30 adhésions. Le nombre de personnes \fs{réellement} r\'eguli\`erement actives dans la vie de l'asso
et ses projets a lui aussi augmenté pour passer à près d'une quinzaine de \fs{personnes} membres.

Au cours des derniers mois, de nombreux projets ont vu le jour, qu'ils soient artistiques, ludiques, musicaux,
techniques... nous y reviendrons \f{dans la section~\ref{ssec:projects} de ce rapport}.

A travers ces nouveaux projets, les dons, et les efforts de tous, le HAUM s'équipe peu à peu :

\begin{itemize}
	\item Un prêt de long terme de la part de Romuald C. a permis de disposer d'un poste à souder de bonne qualité et
		d'une broche pour la fraiseuse. Cela s'ajoute à la longue liste de matériel qu'il a prêté et donné ces dernières
		années. Merci à lui !
	\item Un excellent investissement de Romain P., Florent T., Romuald C. a mené la mise en service d'une plastifieuse
		capable de transfèrer du toner et donc à la réalisation de circuits imprimés en interne. Merci à eux !
	\item Les dons des uns et des autres a permis au HAUM de se munir d'une scie à chantourner, d'un pistolet à colle et
		de petit outillage.
	\item Les dons de matériel portent aussi sur l'infrastructure serveur et réseau. Grâce à Romuald C., LinuxMaine,
		Florent T., Jérôme B., Sébastien V. et Mathieu G. il est désormais possible d'écouter de la musique jouée depuis un
		RaspberryPi sur une chaine au local, les musiques sont librement accessibles sur le disque dur donné par Laurent
		V.. Il est aussi possible de remonter jusqu'au serveur du HAUM sur lequel est hébergé le wiki et bientôt le
		reste de l'infrastructure web. Merci à tout ceux qui se sont impliqués dans la mise en place du réseau et merci
		à Augustin de Laveaucoupet (ATC-IT) de nous laisser utiliser sa connexion en point d'entrée/sortie.
\end{itemize}

Tout au long de l'année, le HAUM a aussi pris part aux RepairCafés, initiative de la ville et des conseils de quartier
pour encourager la réparation des objets plutôt que leur remplacement.

Enfin, \fs{en janvier,} le HAUM a de nouveau pris part aux 24h du Code \f{en janvier} en proposant de nouveau un sujet. Une semaine plus
tard, l'association organisait un lieu de jam pour la GlobalGameJam (ouvert seulement aux membres).

\subsubsection{Partenariats et Réseau}

Les partenariats du HAUM en 2015 sont dans la juste ligne de ceux des années précédentes. Dans le cadre des
RepairCafés, le HAUM a pu ainsi travailler avec Cyclamaine, Les Petits Débrouillards, la ville du Mans \f{(Q: via Le Mans M\'etropole?)}
et les Conseils de Quartiers.

Beaucoup de discussions ont eu cours avec Créalab et OrganicOrchestra dans les derniers mois et l'association Teriaki
nous a retrouvé cette année encore sur le festival.

Merci à tous ceux qui nous ont fait confiance et ceux avec qui nous avons pu travailler.

\subsubsection{Communication}

Le HAUM a amélioré son kit de communication à l'occasion de Festival D, à Nantes. Pour l'occasion, des pochettes aux
couleurs et logo de l'association ont été réalisées ainsi que des feuillets à y glisser. L'objectif derrière ce projet
est de disposé d'un outil de communication modulable pour correspondre à l'évènement qui soit esthétiquement réussi.

\subsubsection{Animations publiques}

Voici une liste des animations publiques auxqelles l'association a participé ou qu'elle a (co-)organisé :

\begin{itemize}
    \item Formations arduino
    \item Mise en place d'une soirée de mini-conférences libres
    \item 24h du Code (partenariat ENSIM / CCI Ruche Numérique)
	\item GlobalGameJam
    \item Festival Teriaki (partenariat Teriaki)
	\item RepairsCafé (4 dates, partenariat Ville du Mans, Conseils de Quartier, Petits Débrouillard, Cyclamaine, etc...)
	\item OpenBidouille Camp (à la fête de quartier Bellevue)
	\item Festival D (Nantes, invitation de Ping)
\end{itemize}

\subsubsection{Projets}
\label{ssec:projects}

Dans les 

\subsubsection{Changement de lieu}

Alors que les mois passent et que le matériel s'entasse dans les 29m\textsuperscript{2} de notre local, il devient
urgent d'envisager un second déménagement (moins précipité que le premier). \fs{Les} Des discussions avec sont en cours entre le bureau et \fs{sur} plusieurs
possibilités : Créalab, Etoc, autre... Beaucoup de choses restent à décider.

\subsection{Objectifs}

Comme l'an dernier, il est urgent d'orienter l'activité de l'association autour de l'information du public, d'envisager
la sensibilisation sur des sujets de société (comme la surveillance de masse, la protection des données personnelles).
Ces points sont importants car ce sont les seuls sur \fs{lequelles} lesquels porter le nom <<~hackerspace~>> permettra d'avoir un
impact sur les gens.
\f{Cela peut se traduire pour l'organisation de nouvelle conf\'erences (a.k.a, les HAUMTalks) r\'eguli\`eres qui ont fait d\'efaut cette ann\'ee. (Q: on voit avec l'Epicerie du Pr\'e pour un partenariat? Une conf. / mois ou tous les deux mois dans leur salle \'equip\'ee d'un \'ecran?)} 

Le second objectif tourne autour de la communication.
Le HAUM doit se doter d'un dossier de communication, d'un dossier
de presse et surtout, il faudra changer la manière de gérer les relations avec les partenaires.
En effet, plusieurs fois cette année, des incompréhensions sont nées de la discussion de plusieurs membres avec un seul
et même partenaire : désigner un membre responsable de la discussion pour chaque structure avec laquelle l'association
travaille permettra surement de clarifier la situation (tant pour nous que pour nos interlocuteurs).\\
\f{Par ailleurs, un certain effort a \'et\'e fait cette ann\'ee autours des comptes-rendus de s\'eances. Cet effort doit se poursuivre pour continuer d'informer les membres, mais \'egalement les personnes ext\'erieures \`a l'association, sur l'avancement des projets et la vie de l'assocation. Ces comptes-rendu \'etant publi\'es sur le site internet, ils participent \`a l'animation et l'enrichissement de ce dernier.} 

\section{Rapport financier}

\subsection{Bilan}

\subsection{Objectifs}


\section{Motions}

\end{document}
