\documentclass[a4paper, 11pt]{article}
	\usepackage[utf8]{inputenc}
    \usepackage[french]{babel}
    \usepackage[T1]{fontenc}
    \usepackage[top=3cm,, right=2cm,bottom=2cm,left=2cm]{geometry}
    \usepackage{eurosym} %pour le symbole euro
    
    \usepackage{tabularx}
    
    \title{Hackerspace Au Mans}
    \author{Procès-Verbal des Assemblées Générales Ordinaire et Extraordinaire}
    \date{Le mardi 17 novembre 2015}
    
    \newcommand\sep{\noindent\rule{\linewidth}{.5pt}}
    
    \newcommand{\vote}[5]{
    
    \smallskip
    \fbox{\begin{minipage}[l]{\textwidth}
    	\smallskip
        \begin{center}
        	\underline{\textsc{Vote}}
        \end{center}
        
        #1\\
        \textbf{Votants} #2\\
        \textbf{Pour} #3\\
        \textbf{Contre} #4\\
        \textbf{NSPP} #5
        
        \smallskip
        
    \end{minipage}}
        \medskip
    }
    
    \newcommand\question[1]{\underline{\textsc{#1}} : }
    
    %\underline{#2}}
    
    
\begin{document}

\maketitle

\section*{Préambule}

Le HAUM remercie la Ruche Numérique pour le prêt de la salle pour cette réunion.

\section{Effectifs}

\begin{itemize}
	\item Présents : 
		\begin{itemize}
		  \item DINSENMEYER Alice
		  \item LECHAT Thomas
		  \item GABORIT Mathieu
		  \item BEN ICHE Mehdi
		  \item BREHERET Jérôme
		  \item PORTE Romain
		  \item VALLÉE Sébastien
		  \item VANNIER Laurent \textsuperscript{2}
		  \item FAVREAU Tifenn
		  \item TOUCHARD Florent
		  \item CUDENNEC Armelle
		  \item LAPLAUD François
		  \item CONTY Romuald
		  \item BRELON Corentin \textsuperscript{1} 
		  \item DUTHE Yannick
		  \item PERRIER Bertrand 
		  \item ROUSSEAU Fabien \textsuperscript{2}
		  \item ??? Gabriel \textsuperscript{2}
		  \item GABORIT Matthieu \textsuperscript{2}
		  \item LEFRANCOIS Jean-Marc \textsuperscript{1} \textsuperscript{2}
		\end{itemize}
	\item Absents ayant donné procuration à BREHERET Jérôme :
		\begin{itemize}
			\item HERISSON Stéphane 
		\end{itemize}
	\item Absent ayant donné procuration à GABORIT Mathieu :
		\begin{itemize}
			\item BLAIN Fred
		\end{itemize}
\item Absent ayant donné procuration à LECHAT Thomas :
		\begin{itemize}
			\item EREAU Baptiste
		\end{itemize}
\item Absent ayant donné procuration à DINSENMEYER Alice :
		\begin{itemize}
			\item DAZEL Olivier
		\end{itemize}
\item Absent ayant donné procuration à CUDENNEC Armelle :
		\begin{itemize}
			\item COSTE Jérôme
			\item PARSHAD Nathalie
		\end{itemize}
\end{itemize}
\textsuperscript{1} Non-membre de l'association\\
\textsuperscript{2} Arrivés à 20h

\bigskip
\textbf{La séance est ouverte à 19h00}

La présidence actuelle décide d'accorder le droit de vote à tous les présents pour la durée de cette AG.

\section{Rapport Moral}

Le rapport moral (cf annexe) est présenté par le président.

\vote{Rapport moral}{21}{21}{0}{0}

Le rapport moral est validé par l'assemblée.

\question{Remarque sur le rapport moral} : 
Les projets peuvent être gérés \textit{via} un "Framaboard". Dans tous les cas, ça ne doit pas remplacer les rapports de séance jugés utiles pour la communication et l'archivage.

\section{Rapport Financier}

Le trésorier présente le rapport financier (cf annexe).

\vote{Rapport Financier (après 20h)}{26}{26}{0}{0}

Le rapport financier est validé par l'assemblée.

Voici le détail du coût approximatif de chaque projet :
\begin{itemize}
	\item Dhaum : 150\euro pour la structure et 50\euro pour sa boîte de rangement
	\item Pong1D : 75\euro
	\item PCBlastifieuse : 50\euro et 20\euro pour les produits chimiques consommables.
\end{itemize}




\subsection{Question et remarques sur le rapport financier}

	\question{Présence au "village de Makers qui sera monté lors de la Gamer Assembly organisée à Poitiers du 25 au 27 mars 2016" ?} L'avis général est d'y participer. Il est évoqué l'idée de demander une contrepartie financière, sous forme d'une aide à l'hébergement et au transport par exemple.\\
	\question{Peut-on utiliser Paypal pour simplifier les factures ?} La trésorière répond que les factures papier sont simples à mettre en oeuvre, et rappelle qu'il faut juste que "HAUM" apparaisse dessus.\\
	\question{Consulte-t-on les membres au sujet du financement de chaque projet par la trésorerie du HAUM ?} Les discussions au cours des séances suffisent pour l'instant à dégager un consensus général sur le financement ou non de chaque projet.\\
	\question{Rachète-t-on une imprimante 3D ?} Ça semble utile pour la communication et attirer de nouveaux membres, mais c'est hors budget. Une solution est de remonter la Tobecca qui n'est plus fonctionnelle.\\
	\question{Où peut-on avoir accès aux détails des finances ?} Sur demande à tout moment auprès de la trésorière.
	


%====================================================================================



\section{Cotisation}

Il est proposé d'augmenter la cotisation pour augmenter le budget alloué aux projets et à l'achat de matériel.

\fbox{\begin{minipage}{\textwidth}
    		\smallskip
       		\begin{center}
        		\underline{\textsc{Vote}}\\
        		Vote pour un montant de 
      	 	\end{center}
        
       		\begin{tabularx}{2\textwidth}{c c c c c c }
25\euro & 30\euro & 35\euro & 40\euro & \textbf{NSPP} & \textbf{Contre}\\
 8 & 13 & 0 & 2 & 3 & 0\\
            %\smallskip        
    		\end{tabularx}
\end{minipage}}
 La cotisation est donc montée à 30\euro.


\section{Questions diverses}

\question{Peut-on avoir accès au local en dehors du mardi ?} Oui, à condition de l'organiser \textit{via} la liste de diffusion mail.\\
\question{Que fait-on aux 24h du Code 2016 ?} Personne ne souhaite proposer une équipe concourante. Le HAUM proposera éventuellement un sujet autour de la lumière si ça lui est demandé rapidement.
\question{Refait-on une GlobalGameJam ? Sera-t-elle publique ?} La GGJ sera organisée au Mans, avec ouverture à quelques invités dans la limite d'une quinzaine de personnes.
\question{Meubles en cartons : Que fait-on du fauteuil qui est volumineux ? Que fait-on des chutes de carton (volumineuses) ?} Les chutes sont à recycler. Le fauteuil reste monté au local.
\question{Matthieu G. dit qu'il serait bon de demander des subventions/financements auprès de la ville, département,... pour les projets à venir.} Il se dit prêt à aider ponctuellement à cette tâche.
\question{Mathieu G. rappelle que des cartons "free to hack" sont toujours libres d'accès.}
\question{Refait-on une galette des rois ?} Non, c'est trop banal et ça n'attire personne. Un autre événement sera organisé pour présenter les projets de l'association autour d'un goûter, avec invitation des partenaires, fin février.



\section{Election du bureau}

Se présentent à l'élection au bureau :

\begin{itemize}
  \item R. CONTY
  \item M. GABORIT
  \item S. VALLÉE
  \item J. BRÉHERET
  \item A. CUDENNEC
  \item M. GABORIT
  \item F. TOUCHARD
  \item L. VANNIER
  \item F. ROUSSEAU
\end{itemize}

Pour ce vote, le collège électoral est réduit au membres seuls (soient 24 votants).
   \hspace{-1cm} \fbox{\begin{minipage}{1.06\textwidth}
    		\smallskip
       		\begin{center}
        		\underline{\textsc{Vote}}\\
        		Élection au bureau de 
      	 	\end{center}
        
       		\begin{tabularx}{2\textwidth}{p{1.6cm} | p{1.6cm} | p{1.6cm} | p{1.6cm} | p{1.6cm} | p{1.6cm} | p{1.6cm} | p{1.6cm} | p{1.6cm} }
\small \textsc{Bréheret} J.& \small \textsc{Rousseau} F.& \small \textsc{Conty} R. &\small \textsc{Vallée} S.&\small \textsc{Cudennec} A.&\small  \textsc{Gaborit} M.&\small \textsc{Gaborit} M. & \small \textsc{Touchard} F. & \small \textsc{Vannier} L.\\     		            \small \textbf{Votants} 24 &  \small \textbf{Votants} 24 &  \small \textbf{Votants} 24 &\small \textbf{Votants} 24& \small \textbf{Votants} 24& \small \textbf{Votants} 24 &\small \textbf{Votants} 24 & \small \textbf{Votants} 24 & \small \textbf{Votants} 24 \\
       		 	\textbf{Pour} 24 & \textbf{Pour} 24&\textbf{Pour} 24&\textbf{Pour} 24&\textbf{Pour} 24&\textbf{Pour} 24&\textbf{Pour} 24 & \textbf{Pour} 24 & \textbf{Pour} 24\\
        		\textbf{Contre} 0&\textbf{Contre} 0&\textbf{Contre} 0&\textbf{Contre} 0&\textbf{Contre} 0&\textbf{Contre} 0&\textbf{Contre} 0 & \textbf{Contre} 0 & \textbf{Contre} 0\\
        		\textbf{NSPP} 0&\textbf{NSPP} 0&\textbf{NSPP} 0&\textbf{NSPP} 0&\textbf{NSPP} 0&\textbf{NSPP} 0&\textbf{NSPP} 0 & \textbf{NSPP} 0 & \textbf{NSPP} 0 \\
            %\smallskip        
    		\end{tabularx}
    		
    		
   
   
    \end{minipage}
    	
    }
        \medskip
        

\section{Réunion de bureau}

Alors que l'ensemble des membres du nouveau bureau sont présents, il est décidé de procéder dans l'immédiat aux élections internes à celui-ci. Pour les votes suivants, le collège électoral est réduit au bureau seul (soient 9 membres).

\vote{S. VALLÉE au poste de secrétaire}{9}{9}{0}{0}

\vote{A. CUDENNEC au poste de trésorière}{9}{9}{0}{0}



\vote{R. CONTY au poste de président}{9}{9}{0}{0}
\vote{M. GABORIT au poste de vice-président}{9}{9}{0}{0}
\vote{J. BRÉHERET au poste de vice-président}{9}{9}{0}{0}

\vote{L. VANNIER au poste de vice-trésorier}{9}{9}{0}{0}
\vote{M. GABORIT au poste de vice-trésorier}{9}{9}{0}{0}

\section{Composition du bureau}
\begin{description}
  \item[Président] R. CONTY
  \item[Vice-Président] M. GABORIT
  \item[Vice-Président]J. BRÉHERET
  \item[Trésorière] Armelle CUDENNEC
  \item[Vice-Trésorier]L. VANNIER
  \item[Vice-Trésorier]M. GABORIT
  \item[Secrétaire] S. VALLÉE
  \item[Membres du bureau] \hfill
  	\begin{itemize}
        \item F. TOUCHARD
        \item F. ROUSSEAU
    \end{itemize}
\end{description}
\bigskip
\textbf{La séance est levée à 21h.}

\bigskip\bigskip

\sep

\bigskip\bigskip

Le présent procès-verbal est approuvé par le président du HAUM.

\bigskip\bigskip

Président :



\newpage

\section*{\textsc{Annexe : Rapports moral et financier}} 


\end{document}