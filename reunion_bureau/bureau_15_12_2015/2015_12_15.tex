\documentclass[a4paper, 11pt]{article}

	\usepackage[utf8]{inputenc}
	\usepackage[T1]{fontenc}
	\usepackage[french]{babel}
	\usepackage{amsmath}

    \title{HAUM}
    \author{Procès Verbal de Réunion de Bureau}
	\date{15 Décembre 2015}


\begin{document}
	\maketitle

	\section{Ordre du Jour}

	\begin{itemize}
		\item Élection d'un nouveau trésorier
		\item Définition d'un quorum
	\end{itemize}

	\section{Présence}

	\subsection{Présents}

	\begin{itemize}
		\item Jérôme \textsc{Bréhéret}
		\item Romuald \textsc{Conty}
		\item Florent \textsc{Touchard}
		\item Fabien \textsc{Rousseau}
		\item Matthieu \textsc{Gaborit}
		\item Laurent \textsc{Vannier}
		\item Mathieu \textsc{Gaborit}
	\end{itemize}

	\subsection{Absents Excusés}

	\begin{itemize}
		\item Armelle \textsc{Cudennec}
		\item Sébastien \textsc{Vallée}
	\end{itemize}

	\section{Réunion}

	\bigskip

	Ouverture de la séance à 20h30.

	\bigskip

	\subsection{Élection du nouveau trésorier}

	Suite à la démission d'Armelle \textsc{Cudennec}, il est nécessaire d'élire un nouveau trésorier au sein du
	bureau.

	Laurent \textsc{Vannier}, trésorier adjoint, se propose pour le poste.

	Laurent \textsc{Vannier} est élu à l'unanimité des présents, il devient donc le nouveau trésorier de
	l'association.

	\section{Définition d'un quorum}
	
	Compte-tenu que le règlement intérieur actuel ne définit par de quorum pour les réunions du bureau, il semble
	nécessaire de palier ce manque. Par consensus, le quorum sera désormais fixé aux deux-tiers des membres du
	bureau.

	\bigskip

	Fin de la séance à 21h.

	\bigskip


\end{document}

