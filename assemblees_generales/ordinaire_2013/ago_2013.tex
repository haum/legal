% vim:ft=tex:
%
\documentclass[11pt]{article}
\usepackage[french]{babel}
\usepackage[T1]{fontenc}
\usepackage[utf8]{inputenc}
\usepackage{amsmath}
\usepackage{url}

\usepackage[top=3cm,right=2cm,bottom=2cm,left=2cm]{geometry}

\title{HAUM}
\author{Assemblée Générale Ordinaire}
\date{12 Novembre 2013}

\begin{document}
\maketitle


\section*{Convocation}

Madame, Monsieur, 

L'association HAUM (loi 1901) vous convoque à ses Assemblées Générales Ordinaire et Extraordinaire qui se tiendront le :

\begin{center}
{\Large 12 novembre 2013 à 19h00}\\
à la Ruche Numérique, 19 Bd Oyon à Le Mans.
\end{center}

En cas d'impossibilité, veuillez prévenir et veiller à vous faire représenter si vous le souhaitez (2 procurations maximum par personne).

\section*{Déroulement}

\begin{enumerate}
    \item Présentation de l'association
    \item Rapport moral
        \begin{enumerate}
            \item Bilan
            \item Objectifs
            \item Questions
        \end{enumerate}
    \item Rapport financier
        \begin{enumerate}
            \item Bilan
            \item Objectifs
            \item Questions
        \end{enumerate}
    \item Motions et vote
    \item Election du nouveau bureau
    \item Questions diverses
\end{enumerate}

\section{Présentation de l'association}
\section{Rapport Moral}

\subsection{Bilan}

L'association HAUM a été créée le 13 novembre 2012.

Comptant initialement 11 adhérents (cf. procès verbal de l'Assemblée Constitutive), elle totalise à date de cette
assemblée générale ordinaire 13 adhérents.

Au cours de l'année 2013, l'association a organisé une Assemblée Générale Extraordinaire (le 29 janvier 2013) pour fixer
la cotisation : celle-ci est désormais de 10 EUR. La cotisation n'a pas été fixée plus tôt pour permettre l'ouverture d'un
compte bancaire au nom du HAUM au Crédit Agricole (agence Place de la République, 72000, Le Mans).

Afin d'être conforme avec la loi en vigueur, l'association a aussi contracté une assurance auprès de la SMACL et ce par
l'intermédaire du Crédit Agricole (plus d'information dans le rapport financier).

L'association s'est aussi munie d'un nom de domaine et a déployé 2 sites sur un serveur appartenant à l'un de ses
membres (sans clause de location). Ces sites sont accessibles aux adressses suivantes :

\begin{description}
    \item[\url{http://www.haum.org}] Site de l'association elle-même (aussi accessible à \url{http://haum.org})
    \item[\url{http://pads.haum.org}] Site de l'instance EtherpadLite\footnote{bloc-notes collaboratifs}
\end{description}

L'association tient à remercier Jérome Bréhéret de lui avoir fourni ce nom de domaine.

Le HAUM a aussi participé à plusieurs évènements autour du développement numérique, du logiciel libre et du mouvement
hacker/maker :

\begin{itemize}
    \item Jellys, organisés par la CCI/Ruche Numérique ;
    \item Ici Les Boîtes Bougent, organisé par le \textsc{Medef} Sarthe ;
    \item Jeudis du libre, organisés par LinuxMaine ;
    \item Samedi Bidouille, co-organisé par LinuxMaine et le HAUM.
\end{itemize}

Enfin, une partie des membres sont allés faire du \textit{coworking} dans les locaux de la Ruche Numérique.

L'association remercie par ailleurs ces différents partenaires pour leurs invitations.

Plusieurs interviews ont eu lieu : SweetFM, RadioAlpa, RCF Le Mans pour les radios ainsi que Ouest France pour la presse.

Plus récemment, le bureau a décidé d'apporter son soutien au projet de Fablab porté par la Ruche Numérique dans le cadre
de l'appel à projet autour des nouveaux EPN\footnote{Espaces Publics Numériques}.

\subsection{Objectifs}

Il serait bon d'essayer d'avoir des jours/horaires d'ouverture fixes d'une semaine sur l'autre pour favoriser la
création d'une communauté.

Il pourrait être intéressant d'essayer de proposer une ouverture le week-end ou de continuer les Samedis Bidouille.

Un effort particulier sera apporté au cours de la prochaine année à la transparence de la tenue de l'association. Le
bureau tient à rappeller à tous que la communication est libre et que tous peuvent proposer des idées visant à améliorer
le fonctionnement de la structure.

Dans ce domaine de la communication, le bureau rappelle aussi que les 3 principaux moyens de communication au sein de
l'association sont :

\begin{itemize}
    \item la mailing-list : haum\_hackerspace@lists.matael.org ;
    \item le canal IRC : \#haum sur Freenode ;
    \item les rencontres IRL (séances ouvertes, jeudis du libre, etc...).
\end{itemize}

Il est possible de créer de nouvelles mailing-lists pour les projets en ayant besoin.

L'association encourage enfin ses membres à partager avec les autres leurs avancées sur les projets \textit{via} la
mailing-list, le canal IRC ou en proposant des pages à ajouter au site web.

Pour ce qui est de la communication externe, nous disposons :

\begin{itemize}
    \item du site ;
    \item des interviews (radio, presse) ;
    \item de la mailing-list ;
    \item d'éventuels sites collaboratifs.
\end{itemize}

\section{Rapport financier}

\subsection{Bilan}

L'AGE du 29 janvier 2013 ayant fixé la cotisation à 10 EUR (toute adhésion d'un montant supérieur étant considérée comme
don), l'association a reçu au total 130 EUR de ses adhérents. Il faut rajouter à cela 15 EUR de dons et le paiement du nom de
domaine par un membre pour l'année en cours. La trésorerie atteste que tous les membres actifs de l'association sont à
jour de cotisation.

Le seul poste de dépense de cette année est la contraction d'une assurance auprès de la SMACL (par l'intermédiare du
Crédit Agricole, nous donnant ainsi accès à une réduction de 50\% pour la première année).
La dépense est donc de 34,29 EUR pour cette année, celle-ci passant environ à 80 EUR pour la suite.

Compte tenu de tout cela, le compte de l'association est aujourd'hui crédité de 110,71 EUR. La trésorerie estime que le
montant actuel de la cotisation est suffisant pour permettre à l'association de vivre et de se développer.

Avec l'état actuel des comptes (même une fois déduites les dépenses inhérentes à une nouvelle année d'exercice), il sera
possible de procéder à certains achats de matériel (au besoin).

\subsection{Objectifs}

Pour la trésorerie, les objectifs de cette nouvelle année seront :

\begin{itemize}
    \item obtenir un accès internet à l'état des comptes,
    \item éventuellement proposer un rapport partiel au bout de 6 mois d'exercice,
    \item maintenir correct l'état de trésorerie durant la durée de l'exercice 2014.
\end{itemize}

\section{Motions}

\subsection{Amendements aux statuts}

\textbf{Rappel :} Tout amendement aux statuts entrainera la re-déposition de ceux ci auprès de la préfecture.

\subsubsection{Statuts des membres}

Il a été demandé par des personnes extérieures s'il existait un statut de membre honoraire, membre sponsor ou membre "moral" (au sens d'une personne morale adhérente).

Pour ce qui est du sponsoring pur, le bureau rappelle que des dons sont toujours possibles.

Pour la création d'un statut de membre honoraire/membre virtuel, il est proposé l'amendement suivant (à la fin de l'article 5) :

\begin{quote}
Est considéré membre virtuel de l'association toute organisation ou personne morale adhérant selon des modalités identiques à celles des membres actifs (précisées ci-avant).

Contrairement aux membres actifs, les membres virtuels ne pourront pas :
{\bfseries
\begin{enumerate}
	\item candidater pour les élections du bureau
    \item voter en assemblée générale ordinaire/extraordinaire
    \item participer aux convocations du bureau par les membres
    \item participer aux votes concernant la fusion/dissolution de l'association
\end{enumerate}
}
\end{quote}

Les propositions 1 à 4 sont à voter séparément dans le courant de l'assemblée générale.

\subsection{Amendements au règlement intérieur}

\subsubsection{Procédure et obligation de réunion pour le bureau}

Au cours de cette première année d'exercice, il est apparu que 3 réunions de bureau n'étaient pas forcément nécessaires. Par conséquent, il est proposé de supprimer cette contrainte. En contrepartie, il est ajouté la possibilité de la convocation d'une réunion du bureau par la majorité des membres actifs (même hors bureau).

Sont donc proposées les modifications suivantes :

\begin{itemize}
	\item Article 5, paragraphe premier : \textit{"Le bureau se réunit au moins \textbf{une} fois par annnée calendaire [...] sur la demande de la moitié de ses membres \textbf{ou de la majorité des membres actifs} [...]"} ;
    \item Article 5, paragraphe 2 : \textit{"[...] avant la date de réunion, \textbf{ainsi qu'à l'ensemble des membres}."}
\end{itemize}

\appendix % annexes

\section*{Lettre de soutien au projet de FabLab}

\begin{quotation}
    Le Hackerspace de l'Université du Maine (HAUM -- association loi 1901) affirme que le projet de fablab porté par la
    Ruche Numérique intéresse ses membres.

    Pour une partie au moins, les objets du HAUM et d'un fablab se rejoignent et ces deux entités seraient en partie
    complémentaires.

    La possibilité de l'accès à une telle structure sera un plus pour les membres du HAUM et permettra de poursuivre la
    sensibilisation du public à de nouveaux outils : informatiques d'une part, mais aussi techniques par l'intermédiaire
    des installations du fablab.

    Les membres du HAUM sont intéressés par le projet et le soutiennent.


\end{quotation}

\end{document}
