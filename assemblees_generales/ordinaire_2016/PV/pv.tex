\documentclass[a4paper, 11pt]{article}
	\usepackage[utf8]{inputenc}
    \usepackage[french]{babel}
    \usepackage[T1]{fontenc}
    \usepackage[top=3cm,, right=2cm,bottom=2cm,left=2cm]{geometry}
    \usepackage{eurosym} %pour le symbole euro

    \usepackage{tabularx}
    \usepackage{soul}

    \title{Hackerspace Au Mans}
    \author{Procès-Verbal des Assemblées Générales Ordinaire et Extraordinaire}
    \date{Le jeudi 10 novembre 2016}

    \newcommand\sep{\noindent\rule{\linewidth}{.5pt}}

    \newcommand{\vote}[5]{

    \smallskip
    \fbox{\begin{minipage}[l]{\textwidth}
    	\smallskip
        \begin{center}
        	\ul{\textsc{Vote}}
        \end{center}

        #1\\
        \textbf{Votants} #2\\
        \textbf{Pour} #3\\
        \textbf{Contre} #4\\
        \textbf{NSPP} #5

        \smallskip

    \end{minipage}}
        \medskip
    }

    \newcommand\question[2]{\noindent\ul{\textit{\textsc{$\bullet$ #1}}}\\#2\\}

    %\ul{#2}}


\begin{document}

\maketitle

\section*{Préambule}

Le HAUM remercie la Ruche Numérique pour le prêt de la salle pour cette réunion.

\section{Effectifs}

\begin{itemize}
	\item Présents :
		\begin{itemize}
      \item CONTY Romuald
      \item FAVREAU Tifenn
      \item BRETON Corentin
      \item PORTE Romain
      \item BEN ICHE Mehdi
      \item TOUCHARD Florent
      \item GABORIT Mathieu
      \item BRÉHÉRET Jérôme
      \item VANNIER Laurent
      \item VALLÉE Sébastien
      \item LEFRANÇOIS Jean-Marc
      \item MAURICE Fabien
      \item BIDAULT Bastien
      \item GABORIT-LEBREQUER Matthieu

		\end{itemize}
	\item Absents ayant donné procuration à CONTY Romuald :
		\begin{itemize}
			\item CHAINOT-BATAILLE Manuella
		\end{itemize}
	\item Absent ayant donné procuration à GABORIT Mathieu :
		\begin{itemize}
			\item BLAIN Fred
      \item DAZEL Olivier
		\end{itemize}
\item Absent ayant donné procuration à BRETON Corentin :
		\begin{itemize}
			\item BATAILLE Fabien
      \item GERVOT Suzy-Lou
		\end{itemize}
\item Absent ayant donné procuration à BRÉHÉRET Jérôme :
		\begin{itemize}
			\item DUTHÉ Yannick
		\end{itemize}
\item Absent ayant donné procuration à TOUCHARD Florent :
		\begin{itemize}
			\item COSTE Jérôme
			\item COFFIGNAL Alexandre
		\end{itemize}
\item Absent ayant donné procuration à PORTE Romain :
		\begin{itemize}
			\item ROUSSEAU Wojciech
		\end{itemize}
\end{itemize}

\bigskip
\textbf{La séance est ouverte à 19h30}

%%La présidence actuelle décide d'accorder le droit de vote à tous les présents pour la durée de cette AG.

\section{Rapport Moral}

Le rapport moral (cf annexe) est présenté par le président.

\vote{Rapport moral}{21}{21}{0}{0}

Le rapport moral est validé par l'assemblée.

\question{Remarque sur le rapport moral}{Les projets peuvent être gérés
\textit{via} un "Framaboard". Dans tous les cas, ça ne doit pas remplacer les rapports de séance jugés utiles pour la communication et l'archivage.}

\section{Rapport Financier}

Le trésorier présente le rapport financier (cf annexe).

\vote{Rapport Financier (après 20h)}{26}{26}{0}{0}

Le rapport financier est validé par l'assemblée.

Voici le détail du coût approximatif de chaque projet :
\begin{itemize}
	\item Dhaum : 150\euro{} pour la structure et 50\euro{} pour sa boîte de rangement
	\item Pong1D : 75\euro{}
	\item PCBlastifieuse : 50\euro{} et 20\euro{} pour les produits chimiques consommables.
\end{itemize}




\subsection{Question et remarques sur le rapport financier}
	\question{Présence au "village de Makers qui sera monté lors de la Gamer
	Assembly organisée à Poitiers du 25 au 27 mars 2016" ?}{L'avis général est d'y participer. Il est évoqué l'idée de demander une contrepartie financière, sous forme d'une aide à l'hébergement et au transport par exemple.}
	\question{Peut-on utiliser Paypal pour simplifier les factures ?}{La trésorière
	répond que les factures papier sont simples à mettre en oeuvre, et rappelle qu'il faut juste que "HAUM" apparaisse dessus.}
	\question{Consulte-t-on les membres au sujet du financement de chaque projet par la trésorerie du HAUM ?}
	{Les discussions au cours des séances suffisent pour l'instant à dégager un
	consensus général sur le financement ou non de chaque projet.}
	\question{Rachète-t-on une imprimante 3D ?}{Ça semble utile pour la
	communication et attirer de nouveaux membres, mais c'est hors budget. Une solution est de remonter la Tobecca qui n'est plus fonctionnelle.}
	\question{Où peut-on avoir accès aux détails des finances ?}{Sur demande à tout
	moment auprès de la trésorière.}



%====================================================================================



\section{Cotisation}

Il est proposé d'augmenter la cotisation pour augmenter le budget alloué aux projets et à l'achat de matériel.

\fbox{\begin{minipage}{\textwidth}
    		\smallskip
       		\begin{center}
        		\ul{\textsc{Vote}}\\
        		Vote pour le montant de la cotisation
      	 	\end{center}

       		\begin{tabularx}{2\textwidth}{c c c c c c }
25\euro{} & 30\euro{} & 35\euro{} & 40\euro{} & \textbf{NSPP} & \textbf{Contre}\\
 8 & 13 & 0 & 2 & 3 & 0\\
            %\smallskip
    		\end{tabularx}
\end{minipage}}
~\\ La cotisation sera donc de 30\euro{} pour l'année 2016.


\section{Questions diverses}

\question{Peut-on avoir accès au local en dehors du mardi ?}{Oui, à condition de
l'organiser \textit{via} la liste de diffusion mail.}
\question{Que fait-on aux 24h du Code 2016 ?}{Personne ne souhaite proposer une équipe concourante. Le HAUM proposera
éventuellement un sujet autour de la lumière si ça lui est demandé rapidement.}
\question{Refait-on une GlobalGameJam ? Sera-t-elle publique ?}{La GGJ sera
organisée au Mans, avec ouverture à quelques invités dans la limite d'une quinzaine de personnes.}
\question{Meubles en cartons : Que fait-on du fauteuil qui est volumineux ? Que
fait-on des chutes de carton (volumineuses) ?}{Les chutes sont à recycler. Le
fauteuil reste monté au local.}
\question{Matthieu G. dit qu'il serait bon de demander des
subventions/financements auprès de la ville, département,... pour les projets
à venir.}{Il se dit prêt à aider ponctuellement à cette tâche.}
\question{Mathieu G. rappelle que des cartons "free to hack" sont toujours libres d'accès.}{Il ne faut surtout pas hésiter à se servir dedans pour recuperer des composants
ou plus.}
\question{Refait-on une galette des rois ?}{Non, c'est trop banal et ça
n'attire personne. Un autre événement sera organisé pour présenter les projets
de l'association autour d'un goûter, avec invitation des partenaires, fin février.}



\section{Election du bureau}

Se présentent à l'élection au bureau :

\begin{itemize}
  \item Romuald CONTY
  \item Mathieu GABORIT
  \item Sébastien VALLÉE
  \item Jérôme BRÉHERET
  \item Armelle CUDENNEC
  \item Matthieu GABORIT
  \item Florent TOUCHARD
  \item Laurent VANNIER
  \item Fabien ROUSSEAU
\end{itemize}

Pour ce vote, le collège électoral est réduit au membres seuls (soient 24 votants).\\
   \hspace{-1cm} \fbox{\begin{minipage}{1.06\textwidth}
    		\smallskip
       		\begin{center}
        		\ul{\textsc{Vote}}\\
        		Élection au bureau de
      	 	\end{center}

       		\begin{tabularx}{2\textwidth}{p{1.6cm} | p{1.6cm} | p{1.6cm} | p{1.6cm} | p{1.6cm} | p{1.6cm} | p{1.6cm} | p{1.6cm} | p{1.6cm} }
\small \textsc{Bréheret} Jérôme& \small \textsc{Rousseau} Fabien& \small \textsc{Conty} Romuald &\small \textsc{Vallée} Sébastien&\small \textsc{Cudennec} Armelle&\small  \textsc{Gaborit} Matthieu&\small \textsc{Gaborit} Mathieu & \small \textsc{Touchard} Florent & \small \textsc{Vannier} Laurent\\     		            \small \textbf{Votants} 24 &  \small \textbf{Votants} 24 &  \small \textbf{Votants} 24 &\small \textbf{Votants} 24& \small \textbf{Votants} 24& \small \textbf{Votants} 24 &\small \textbf{Votants} 24 & \small \textbf{Votants} 24 & \small \textbf{Votants} 24 \\
       		 	\textbf{Pour} 24 & \textbf{Pour} 24&\textbf{Pour} 24&\textbf{Pour} 24&\textbf{Pour} 24&\textbf{Pour} 24&\textbf{Pour} 24 & \textbf{Pour} 24 & \textbf{Pour} 24\\
        		\textbf{Contre} 0&\textbf{Contre} 0&\textbf{Contre} 0&\textbf{Contre} 0&\textbf{Contre} 0&\textbf{Contre} 0&\textbf{Contre} 0 & \textbf{Contre} 0 & \textbf{Contre} 0\\
        		\textbf{NSPP} 0&\textbf{NSPP} 0&\textbf{NSPP} 0&\textbf{NSPP} 0&\textbf{NSPP} 0&\textbf{NSPP} 0&\textbf{NSPP} 0 & \textbf{NSPP} 0 & \textbf{NSPP} 0 \\
            %\smallskip
    		\end{tabularx}




    \end{minipage}

    }
        \medskip


\section{Réunion de bureau}

Alors que l'ensemble des membres du nouveau bureau sont présents, il est décidé de procéder dans l'immédiat aux élections internes à celui-ci. Pour les votes suivants, le collège électoral est réduit au bureau seul (soient 9 membres).

\vote{Romuald CONTY au poste de président}{9}{9}{0}{0}
\vote{Mathieu GABORIT au poste de vice-président}{9}{9}{0}{0}
\vote{Jérôme BRÉHERET au poste de vice-président}{9}{9}{0}{0}

\vote{Armelle CUDENNEC au poste de trésorière}{9}{9}{0}{0}
\vote{Laurent VANNIER au poste de vice-trésorier}{9}{9}{0}{0}
\vote{Matthieu GABORIT au poste de vice-trésorier}{9}{9}{0}{0}

\vote{Sébastien VALLÉE au poste de secrétaire}{9}{9}{0}{0}

\section{Composition du bureau}
\begin{description}
  \item[Président] Romuald CONTY
  \item[Vice-Président] Mathieu GABORIT
  \item[Vice-Président]Jérôme BRÉHERET
  \item[Trésorière] Armelle CUDENNEC
  \item[Vice-Trésorier]Laurent VANNIER
  \item[Vice-Trésorier]Matthieu GABORIT
  \item[Secrétaire] Sébastien VALLÉE
  \item[Membres du bureau] \hfill
  	\begin{itemize}
        \item Florent TOUCHARD
        \item Fabien ROUSSEAU
    \end{itemize}
\end{description}
\bigskip
\textbf{La séance est levée à 21h.}

\bigskip\bigskip

\sep

\bigskip\bigskip

Le présent procès-verbal est approuvé par le président du HAUM.

\bigskip\bigskip

Président :



\newpage

\section*{\textsc{Annexe : Rapports moral et financier}}
\section{Rapport Moral}


\subsection{Bilan}

On a coutume de dire que la première année est un test, la deuxième l'épreuve réelle et la troisième le début de l'age
d'or... Si le HAUM continue sur sa lancée, cet adage pourrait se vérifier.

Cette troisième année d'exercice a été éprouvante pour l'association mais cela semble porter ses fruits et les résultats
commencent à apparaître.

En 2015, le HAUM est passé de 23 à plus de 30 adhésions. Le nombre de personnes régulièrement actives dans la vie de l'asso
et ses projets a, lui aussi, augmenté pour passer à près d'une quinzaine de membres.

Au cours des derniers mois, de nombreux projets ont vu le jour, qu'ils soient artistiques, ludiques, musicaux,
techniques... nous y reviendrons dans la section~\ref{ssec:projects} de ce rapport.

A travers ces nouveaux projets, les dons, et les efforts de tous, le HAUM s'équipe peu à peu :

\begin{itemize}
	\item Un prêt de long terme de la part de Romuald C. a permis de disposer d'un poste à souder de bonne qualité et
		d'une broche pour la fraiseuse. Cela s'ajoute à la longue liste de matériel qu'il a prêté et donné ces dernières
		années. Merci à lui ! L'achat d'un ou plusieurs station de soudage est en discussion.
	\item Un excellent investissement de Romain P., Florent T., Romuald C. a mené à la mise en service d'une plastifieuse
		capable de transfèrer du toner et donc de réaliser des circuits imprimés en interne. Merci à eux !
	\item Les dons des uns et des autres a permis au HAUM de se munir d'une scie à chantourner, d'un pistolet à colle et
		de petit outillage.
	\item Les dons de matériel portent aussi sur l'infrastructure serveur et réseau. Grâce à Romuald C., LinuxMaine,
		Florent T., Jérôme B., Sébastien V. et Mathieu G. il est désormais possible d'écouter de la musique jouée depuis un
		RaspberryPi sur une chaine au local, chacun peut rajouter des musiques sur le disque dur donné par Laurent
		V.. Il est aussi possible de remonter jusqu'au serveur du HAUM sur lequel est hébergé le wiki et bientôt le
		reste de l'infrastructure web. Merci à tout ceux qui se sont impliqués dans la mise en place du réseau et merci
		à Augustin de Laveaucoupet (ATC-IT) de nous laisser utiliser sa connexion en point d'entrée/sortie.
\end{itemize}

Tout au long de l'année, le HAUM a également participé à des animations publiques ( voir section \ref{animationspubliques}).


\subsubsection{Partenariats et Réseau}

Les partenariats du HAUM en 2015 sont dans la juste ligne de ceux des années précédentes. Dans le cadre des
RepairCafés, le HAUM a pu ainsi travailler avec Cyclamaine, Les Petits Débrouillards, la ville du Mans et les Conseils de Quartiers.

Beaucoup de discussions ont eu cours avec Créalab et OrganicOrchestra dans les derniers mois et l'association Teriaki
nous a retrouvé cette année encore sur le festival.

Merci à tous ceux qui nous ont fait confiance et ceux avec qui nous avons pu travailler.

\subsubsection{Communication}

Le HAUM a amélioré son kit de communication à l'occasion de Festival D, à Nantes. Pour l'occasion, des pochettes aux
couleurs et logo de l'association ont été réalisées ainsi que des feuillets à y glisser. L'objectif derrière ce projet
est de disposer d'un outil de communication modulable pour correspondre à l'évènement et que celui ci soit esthétiquement réussi.

Le site web évolue aussi régulièrement, ainsi que le compte Flickr. Merci à tous ceux qui contribuent à le rendre vivant et à l'améliorer.

\subsubsection{Animations publiques \label{animationspubliques}}

Voici une liste des animations publiques auxqelles l'association a participé ou qu'elle a (co-)organisé :

\begin{description}
    \item[Formations arduino]
    \item[Mise en place d'une soirée de mini-conférences libres] la dernière a eu lieu l'hiver dernier et a attiré une
		douzaine de personnes. Il n'y a pas de séance prévue à venir, faute de talkers, d'un site fonctionnel et de
		temps pour les organiser. L'épicerie du pré s'est dite prête à accueillir ce genre d'évènement dans son ex-bibliothèque. Ce partenariat permettrait éventuellement d'attirer plus de monde (public et talkers).
    \item[24h du Code] en partenariat avec l'ENSIM et la CCI Le Mans Sarthe/Ruche Numérique, en 2015, le HAUM a de nouveau proposé un sujet.
	\item[GlobalGameJam] le HAUM a ouvert un lieu de jam à la Ruche Numérique pour les membres du hacklab seulement. L'expérience sera renouvelée en 2016si quelques membres se proposent pour son organisation.
    \item[Festival Teriaki] en partenariat avec Teriaki, le HAUM a pu présenter dHAUMidi, un instrument de musique à taille humaine.
	\item[RepairCafés] Sur 4 dates, en partenariat avec la Ville du Mans, les Conseils de Quartier, les Petits
		Débrouillard, Cyclamaine, etc...
	\item[OpenBidouille Camp] à la fête de quartier Bellevue
	\item[Festival D] à Nantes, sur invitation de Ping.
\end{description}

\subsubsection{Projets}
\label{ssec:projects}

\begin{description}
	\item[HAUMtinsel] Une évolution de la guirlande connectée est en cours de développement pour ce Noël.
	\item[dHAUM/dHAUMidi] Un instrument de musique à base de dôme géodésique présenté dans le cadre du festival Teriaki.
	\item[PCBlastifieuse] Un hack de plastifieuse maintenant PCB-ready.
	\item[Meubles en carton] Un meuble pour l'installation son et réseau, ainsi qu'un rocking chair ont été réalisés.
	\item[Mise à jour du Pong/PolychrHAUM] Écriture et utilisation d'une lib permettant la gestion d'animations sur une bande de LED.
	\item[Rénovation du Pianostairs]
\end{description}

\subsubsection{Changement de lieu}

Alors que les mois passent et que le matériel s'entasse dans les 29m\textsuperscript{2} de notre local, il devient
urgent d'envisager un second déménagement (moins précipité que le premier). Des discussions sont en cours au sein du bureau et plusieurs
possibilités sont envisagées : Créalab, Etoc, autre... Beaucoup de choses restent à décider.

À savoir toutefois : la majorité de ces options implique le paiement d'un loyer ce qui est difficile compte tenu des
finances de l'association.

\subsection{Objectifs}

Le HAUM a progressé dans sa communication avec le grand public \textit{via} les événements sus-cités. Le HAUM pourrait cependant contribuer davantage à la sensibilisation du public sur
des sujets de société comme la surveillance de masse, la protection des données personnelles, \ldots Cela peut se faire
par exemple à travers les talks ou des conférences plus longues.

\medskip

Le HAUM doit aussi se doter d'un dossier de communication, d'un dossier de presse et surtout, il faudra changer la manière de gérer les relations avec les partenaires.
En effet, plusieurs fois cette année, des incompréhensions sont nées de la discussion de plusieurs membres avec un seul
et même partenaire : désigner un membre responsable de la discussion pour chaque structure avec laquelle l'association
travaille permettra surement de clarifier la situation (tant pour nous que pour nos interlocuteurs).

Par ailleurs, un certain effort a été fait cette année autour des compte-rendus de séances. Cet effort doit se
poursuivre pour continuer d'informer les membres, mais également les personnes extérieures à l'association, sur
l'avancement des projets et la vie de l'assocation. Ces comptes-rendu étant publiés sur le site internet, ils
participent à l'animation et l'enrichissement de ce dernier.

\section{Rapport financier}

\subsection{Bilan}

\begin{figure}[!ht]\centering
	\begin{tabular}{c|ccc}
		Objet & Dépenses & Recettes & Soldes (positifs)\\\hline
		Cotisations reçues & & 590 & \\
		Services (impression 3D) & & 5 & \\
		Dons en espèces & & 83 & \\
		Dons en nature & & \parbox[c]{5cm}{Consommables (Makershop),\\outils \& matières premières}  & \\
		Mise à disposition gratuite & & Locaux (Ruche Numérique) & \\\hline\hline
		Achats fournitures & 326.94 & & \\
		Assurance & 86.79 & & \\
		Charge exceptionnelles & 93.15 & & \\\hline\hline
		\textsc{\textbf{Totaux}} & 506.88 & 678 & 171.12\\\hline
		Reliquat 2014 & & & 235.05\\
		Montant disponible & & & 406.17
	\end{tabular}
	\caption{Budget 2015 du HAUM}
\end{figure}

\begin{figure}[!ht]\centering
\begin{tabular}{c|c|cc|c}
	Année & Nombre d'Adhérents & Dépenses & Recettes & Résidu\\\hline
	2013 & 13 & 34.29 & 145 & 110.71\\
	2014 & 23 & 205.66 & 330 & 124.34\\
	2015 & 32 & 506.88 & 678 & 171.12
\end{tabular}
\caption{Evolution du nombre d'adhérents, des dépenses et des recettes.}
\end{figure}

\subsection{Objectifs}

Il apparaît au terme de cette troisième année d'exercice que la trésorerie se porte bien. Certains points restent
toutefois à améliorer.

Le fonctionnement interne du HAUM et la bonne volontée de ses membres font que les projets et l'équipement sont
généralement financés \textit{via} des dons ou des prêts. Cette situation n'est pas viable à long terme et doit cesser :
elle est en effet une source potentielle de conflits et de dissensions internes. L'objectif de l'association, à sa
création, était de permettre la mutualisation des moyens : l'activité financière doit aller en ce sens.

Au cours d'un certain nombre de projets il est apparu que la trésorerie était un sujet sensible et que les cotisations
ne suffisaient pas au financement des projets à long terme. Les matières premières et les outils restent encore trop
souvent la charge de quelques membres.

Plusieurs possibilités s'offrent alors :

\begin{description}
	\item[Augmentation de la cotisation] A 20\euro{}/an, la cotisation du HAUM reste encore une des plus faible en France. La
		possibilité d'augmenter la cotisation n'est pas à écarter.
	\item[Double cotisation] Certains mettent en place une double cotisation, la première permettant d'accèder au
		hackerspace dans le cadre de certains temps bien définis et la seconde permettant d'accèder au lieu sur des
		plages beaucoup plus étendues. A réfléchir.
	\item[Goodies] La vente de goodies est parfois intéressante mais l'expérience menée avec les mugs montrent que cela
		n'intéresse en fait que quelques membres. On peut retenter avec des tee-shirts qui intéresseront peut être plus
		largement.
	\item[Vente d'électronique] Beaucoup de hackerspaces se financent en vendant des cartes permettant de reproduire
		certains de leurs projets ou des kits... peut-être est-il possible de mettre ça en place.
	\item[Subventions/Sponsors par projets] Plus simples à obtenir que des subventions en général, les subventions sur
		projet (ou les sponsoring) permettent de réaliser un projet parfois coûteux sans saigner la trésorerie aux
		quatres veines. Cette option est soumise à la condition que les comptes et la gestion soient
		irréprochables...
\end{description}

Quoi qu'il en soit, un certains nombre de règles devront maintenant êtres appliquées pour faciliter et clarifier la
gestion de la trésorerie :

\begin{itemize}
	\item Toutes les factures doivent être établies \textbf{au nom du HAUM} et ne contenir que des éléments
		\textbf{achetés pour le HAUM}.
	\item Les dépenses de l'argent du hackerspace ne doivent être faites qu'après \textbf{consultation des autres
		membres} et, surtout, \textbf{vérification de la trésorerie}.
\end{itemize}

La question se pose une fois encore sur la possibilité de prioriser le financement des projets. Des discussions sont à
lancer de ce côté là.

Enfin, il faudra, au cours de la prochaine année d'exercice établir une convention d'occupation des lieux en bonne et dûe
forme auprès de la Ruche Numérique.
\end{document}
