\documentclass[11pt,twosided]{article}
\usepackage[french]{babel}
\usepackage[T1]{fontenc}
\usepackage{fontspec}
\usepackage[utf8]{inputenc}
\usepackage{url}
\usepackage{eurosym}

\usepackage{ulem} % to use a strikeout/strikethrough font
\usepackage{color}
\newcommand{\fs}[1]{\textcolor{red}{\sout{#1}}}
\newcommand{\f}[1]{\textcolor{blue}{#1}}

\usepackage[top=3cm,right=2cm,bottom=2cm,left=2cm]{geometry}

\title{HAUM}
\author{Assemblées Générales Ordinaire \& Extraordinaire}
\date{17 Novembre 2015}

\begin{document}
\maketitle


\section*{Convocation}

Madame, Monsieur,

L'association HAUM vous convoque à ses Assemblées Générales Ordinaire \& Extraordinaire qui se tiendront le :

\begin{center}
{\Large 19 Novembre 2016 à 19h00}\\
à la Ruche Numérique,\\19 Bd M\&A Oyon au 1\textsuperscript{er} étage,\\72 100 Le Mans
\end{center}

En cas d'impossibilité, veillez vous faire excuser et vous faire représenter si vous le souhaitez (2 procurations maximum par personne présente, Statuts, art. 6).

\newpage

\vspace{1.5cm}

\hrule
\begin{center}
\Large\bfseries Assemblée Générale Extraordinaire
\end{center}
\hrule

\vspace{1.5cm}

\section*{Déroulement}

\begin{enumerate}
    \item Amendement aux statuts sur la vente de biens (portée par R.Conty \& M. Gaborit)
    \begin{itemize}
        \item Exposé
        \item Débat
        \item Vote
    \end{itemize}
\end{enumerate}


\section{Amendement aux statuts sur la vente de biens}

L'amendement est porté par R. Conty et M. Gaborit.

\subsection{Exposé}

Durant l'année passée, le HAUM a été amené à vendre des biens lui appartenant, soit des éléments d'objets démontés (vendus pour pièces) ou plus récemment des produits de l'activité (Laumios). Cette vente n'est légale qu'à la condition que les statuts reflètent cette activité. Cette amendement vise à mettre l'association en conformité avec la loi.

Il est proposé l'ajout d'un élément à l'objet social (Art. 2 des statuts), ainsi rédigé :

\begin{quotation}
\itshape vendre ou louer des biens appartenant à ou produits par l'association afin de financer des actions futures ou l'achat de matériel.
\end{quotation}

\subsection{Débat}
\subsection{Vote}

\section*{Clotûre}

\pagebreak

\vspace{1.5cm}

\hrule
\begin{center}
\Large\bfseries Assemblée Générale Ordinaire
\end{center}
\hrule

\vspace{1.5cm}

\section*{Déroulement}

\begin{enumerate}
    \item Appel et annonce des procurations
    \item Présentation de l'association
    \item Rapport moral
        \begin{enumerate}
            \item Bilan
            \item Objectifs
            \item Questions
        \end{enumerate}
    \item Rapport financier
        \begin{enumerate}
            \item Bilan
            \item Objectifs
            \item Questions
        \end{enumerate}
    \item Motions et vote
    \item Election du nouveau bureau
    \item Questions diverses
\end{enumerate}

\section{Présentation de l'association}

% TODO

\section{Rapport Moral}

\subsection{Bilan}

Cette année a été riche en changements et en actions menées pour et par l'association.

\subsubsection{Changement de local}

Après près de 2 ans dans nos anciens locaux au sein de la pépinière de la Ruche Numérique, le HAUM commençait à se sentir à l'étroit. Pour que les séances se déroulent dans les meilleures conditions de confort et de sécurité, il était temps de changer de local.

Cette envie a pu se concrétiser grâce au concours de Le Mans Développement, agence de développement économique de la métropole, qui a mis a notre disposition un local 2 fois et demi plus grand (passant donc de 29m\textsuperscript{2} à plus de 70m\textsuperscript{2}). Le Mans Developpement a également réalisé les travaux nécessaires à l'installation de l'association et nous a mis en relation avec les services techniques de la ville du Mans pour que nous puissions récupérer du mobilier mis au rebut et ainsi meubler le lieu.

L'établi du HAUM est toujours mis à sa disposition par la Ruche Numérique malgré notre départ de leurs locaux.

Le bureau tient à remercier tous les acteurs institutionnels et les bénévoles qui ont rendu possible ce déménagement.

\subsubsection{Évolution des adhésions}

TODO

\subsubsection{Participation au projet French Tech}

Les discussions pour le changement de local font partie d'un projet plus vaste de création d'un pôle d'activité sur le secteur de la Gare dans le cadre du label French Tech. L'objectif est de mettre en oeuvre une Cité de l'Innovation Colaborative (CICO) rassemblant un certain nombre d'acteur. L'inclusion du HAUM au projet (en tout cas sa présence sur le lieu de l'éventuelle CICO) a permis au Mans d'obtenir le Label French Tech.

Dans le cadre de ce projet, les discussions portent à la fois sur la réhabilitation de l'ancienne hôpital spécialisé Étoc Demazy mais également sur une "première phase" à plus court terme qui prendrait place dans les locaux de l'ancienne DDCS (boulevard Lefaucheux). Cet aménagement permettrait de rassembler un incubateur, une pépinière et le hackerspace dans 1000m\textsuperscript{2}.

Implications ? Déménagement !

\subsubsection{Partenariats et Réseau}

Dans le cadre des différentes manifestations auxquelles il a participé et des différents projets auxquels il prend part, le HAUM continue de développer un réseau d'acteurs et de partenaires proches. Parmis eux : Teriaki, le Hangar Créalab, le CESI, la Ruche Numérique, l'ENSIM, la CCI du Mans et de la Sarthe, Le Mans Développement, Sarthe Développement, Le Mans Métropole, etc...

Merci à eux.

Gamers Assembly ?

\subsubsection{Animations publiques}

Comme à l'accoutumée, le HAUM a contribué \& organisé en 2016, plusieurs animations à destination du public.

\paragraph{24H du code} Pour l'édition 2016 des 24H du Code (co-organisées par l'ENSIM et la Ruche Numérique), le HAUM a proposé un sujet. Le sujet portait sur la création d'un outil pour explorer des données fournies par Sarthe Développement (ouverture des lieux touristiques). Plusieurs membres du HAUM ont ainsi collaborer pour rédiger le sujet mais aussi être présent toute la nuit aux 24h du code pour guider les équipes.

\paragraph{Bienvenus sur Mars} Organisé au prieuré de Vivoin, le festival BienVenus sur Mars a invité le HAUM a présenter son Pong1D, le dhAUM mais également une création originale. Les Laumios sont nés ainsi et ont engendré de nombreuses idées de projets futurs. La vente de laumios n'est pas exclue et l'affaire est à suivre.

\paragraph{Gamer's Assembly} La Gamers Assembly accueille depuis 2 ans un village maker. Cette année, l'organisation a convié le HAUM a y présenter quelques réalisations. Ce fut l'occasion pour certains de nos membres d'aller pour la première fois sur une aussi grosse manifestation.

\paragraph{Siestes Teriaki} Après les siestes il y a deux ans et le festival l'an dernier, le HAUM est revenu à Teriaki cette année avec le projet Lampes Orbitales (présenté à Vivoin), un mini-spectacles sur Laumios et un labyrinthe géant. L'organisation de l'évènement a posé de nombreux problèmes de gestion du temps et la question se pose désormais : doit on continuer à prendre part à des manifestations demandant autant de travail ?

\paragraph{Le sHAUM} Suite à son déménagement, le hackerspace a invité ses partenaires à venir découvrir le nouveau lieu. Ce fut une bonne occasion pour discuter avec toutes et tous et de faire découvrir l'association et ses possiblités.

\paragraph{Les HAUMTalks} L'association a organisé 6 sessions de ses mini-conférences en 2016 et au moins une septième est prévue. Les HAUMTalks intéressent un public assez large et permettent à chacun de présenter des sujets lui tenant à coeur (hacker). Ces rencontres permettent aussi de faire connaître l'association auprès d'extérieurs.

\subsubsection{Projets}

Parmis les très nombreux travaux menés en séance, nous retiendront tout particulièrement :

\begin{itemize}
    \item Lampes orbitales/Laumio
    \item LaumioAnimator
    \item Labyrinthe Initiati
    \item Amélioration de la fraiseuse
    \item Stations de soudage
    \item sHAUM
    \item Amélioration du 1DPong (lors de la Gamer's Assembly)
    \item Reconstruction software du PianoStairs
\end{itemize}

Ces différents projet permettent de faire vivre et connaître l'association. Merci à tous ceux qui y ont pris part en les construisant, en les documentant, en les hackant et en les faisant évoluer.

\subsubsection{Communication}

Depuis sa création officielle en 2012, le HAUM souffre de problèmes chroniques de communication. En effet, qu'elle soit à destination des membres ou des extérieurs, la communication du HAUM reste spécifique et non-inclusive.

À toutes fins utiles, une partie du bureau tient à rappeler que le moyen de communication officiel de l'association est la liste de discussion par mail.

Les problèmes de communication ont notamment conduits à des malentendus avec nos partenaires, des retards dans la propagation de l'information et des soucis d'organisation plus larges. À l'échelle du hackerspace, la communication trop sporadique entre les membres a privé certains d'opportunités intéressantes et mené certains à se détacher de l'association ou en tout cas des rôles qu'ils y occupaient.

À un autre égard, le manque de réponse sur l'adresse contact cause de graves lenteurs dans nos rapports aux autres et renvoie une image très statique du HAUM.

À l'avenir, il pourrait être intéressant de désigner 2 personnes responsables de \texttt{contact@haum.org} (tout en étendant la diffusion pour information au bureau) de même qu'il est nécessaire d'améliorer rapidement la communication des informations au sein de l'associations. Les personnes en charge des relations avec les partenaires extérieurs \textbf{doivent} s'acquitter du transfert de l'information vers la mailing-liste (et pas uniquement en séance). Les informations pouvant être confidentielles doivent également être transférées a minima sur la liste du bureau, sinon en réunion de bureau.

La vie du hackerspace est une affaire commune, ses relations vers l'extérieur également.

\subsection{Objectifs}


\section{Rapport financier}
\subsection{Bilan}
\subsection{Objectifs}

\section{Motions}
Aucun amendement aux Statuts ou bien au Règlement Intérieur n'a été proposé.

\section{Questions Diverses}

\end{document}
