\documentclass[11pt]{article}
\usepackage[french]{babel}
\usepackage[T1]{fontenc}
\usepackage[utf8]{inputenc}
\usepackage{amsmath}
\usepackage{url}

\usepackage[top=3cm,right=2cm,bottom=2cm,left=2cm]{geometry}

\title{HAUM}
\author{Assemblée Générale Ordinaire}
\date{10 Octobre 2014}

\begin{document}
\maketitle


\section*{Convocation}

Madame, Monsieur, 

L'association HAUM (loi 1901) vous convoque à ses Assemblées Générales Ordinaire et Extraordinaire qui se tiendront le :

\begin{center}
{\Large Vendredi 10 Octobre 2014 à 18h30}\\
à la Ruche Numérique, 19 Bd M\&A Oyon (Gare Sud), 72 000 Le Mans
\end{center}

En cas d'impossibilité, veuillez prévenir et veiller à vous faire représenter si vous le souhaitez (2 procurations maximum par personne).

\section*{Déroulement}

\begin{enumerate}
    \item Présentation de l'association
    \item Rapport moral
        \begin{enumerate}
            \item Bilan
            \item Objectifs
            \item Questions
        \end{enumerate}
    \item Rapport financier
        \begin{enumerate}
            \item Bilan
            \item Objectifs
            \item Questions
        \end{enumerate}
    \item Motions et vote
        \begin{itemize}
            \item Amendements au Règlement Intérieur
			\begin{itemize}
				\item Ré-évaluation de la cotisation
			\end{itemize}
		\end{itemize}
    \item Election du nouveau bureau
    \item Questions diverses
\end{enumerate}

\section{Présentation de l'association}
\section{Rapport Moral}


\subsection{Bilan}

Après une première année d'exercice basée sur la stabilisation de l'association (recherche d'un lieu, d'un rythme, etc...), cette seconde année a été marquée par de nombreux partenariats et une meilleure ouverture vers l'extérieur.

\subsubsection{Changement de lieu}

Le premier élément important de cette rétrospective est évidemment le changement de lieu.
Après un point problématique concernant la mise à disposition du local par l'ISMANS pendant l'été, le HAUM s'est déplacé (sur la proposition de J-M Laffay et L. Richer) vers le futur fablab de la Ruche Numérique.

L'association tient désormais son local au 19 Bd Marie et Alexandre Oyon. Ce nouveau lieu (s'il s'éloigne du berceau universitaire du HAUM) a pour avantage d'être plus facilement accessible en transport en commun et à pied (centre ville). Il présente aussi l'avantage (considérable) de pouvoir être équipé et véritablement transformé en hackerspace (non, une armoire n'est pas suffisante). Enfin, les horaires d'ouverture (potentiellement H-24) en font un lieu bien mieux adapté que l'ISMANS.

L'équipement du nouveau lieu s'est appuyé à la fois sur des dons matériels (notamment du Département Informatique de l'Université du Maine), mais aussi sur des dons et des prêts de la part de plusieurs membres. Merci donc à tous ceux qui ont laissé un peu de leur matériel personnel : Romuald C., Jérôme B., Florent T., Mathieu G.

Le premier de cette liste de contributeurs a aussi placé trois serveurs et un SAN au local. Un grand merci à lui.

Merci aussi à un donateur bienveillant qui nous a offert un four à refusion (et qui ne souhaite pas être associé à ce don).

\subsubsection{Changement d'acronyme}

L'association est déposée sous le nom "HAUM". Après le changement de lieu et de nombreuses remarques (souvent péjoratives) à propos de notre statut d'association étudiante, il a été choisi de changer le nom de l'association (ou plutôt la signification de l'acronyme).

Après un sondage sur la mailing-list, la signification du sigle HAUM passe de "Hackerspace de l'Université du Maine" à "Hackerspace au Mans".

L'objectif principal de ce changement de signification est de se défaire de l'image d'association étudiante (parfois nuisible). Un intérêt secondaire est d'assurer une forme de cohérence entre le nom du hackerspace et le lieu ou il se trouve.

\subsubsection{Partenariats et Réseau}


Côté partenariats ensuite, le HAUM a continué à travailler avec son contact "historique" : la Ruche Numérique. Cette année, cette collaboration s'est étendue de bien des manières :

\begin{itemize}
    \item participation multiple aux animations de la Ruche Numérique (Jellys, rencontres, journées thématiques, etc.) ;
    \item rapprochement géographique : le nouveau local du hackerspace est dans la pépinère ;
    \item co-organisation d'évènements et d'animations : hackathon, rencontres opendata, formations Arduino, etc. ;
    \item animation du fablab en devenir, le Beelab.
\end{itemize}

Cette nouvelle année de bonne entente et de réussite montre que l'association tacite entre la Ruche Numérique et le HAUM était un choix cohérent et heureux.

Dans le cadre du changement de local, l'association a été amenée à travailler avec MakerShop. Ils nous ont prêté plusieurs machines (i.e. imprimantes 3D) et fait don d'une quantité non négligeable de consommables et nous les en remercions. Nous avons aussi été présents avec eux aux 3D Days (co-organisés avec la Ruche Numérique) et assuré une partie de leur promotion au cours de Sarthe Le Mans Connection.

Dans une autre catégorie, le HAUM a été amené, au cours de l'année, à travailler plusieurs fois avec les Petits Débrouillards. La collaboration a pris la forme de soirées autour de la MakeyMakey (organisées à la Ruche), de formation d'animateurs des Petits Débrouillards à l'utilisation de la plateforme Arduino et d'une invitation (de leur part) à participer aux Siestes Teriaki (à l'Abbaye de l'Epau).

La rencontre et la collaboration avec les Petits Débrouillards est une bonne chose et un moyen pour notre association de toucher un autre public.

Dans le même esprit (et au travers du fablab) le HAUM a pu travailler aussi avec les Francas : réparation et amélioration de leur imprimante 3D et co-participation au Forum Jeunes (pôle numérique).

Finalement, quelques membres du HAUM ont pu se déplacer jusqu'a Paris pour visiter d'autres labs. Cette soirée nous a permis de rencontrer notament l'équipe du /tmp/lab et de l'Electrolab et nous a donné plusieurs idées pour la suite.
    
\subsubsection{Communication}

A la moitié de l'année d'exercice environ, le constat concernant la communication du HAUM n'était pas brillant : un compte twitter réservé aux annonces d'ouverture, un site peu ergonomique et du contenu sans intérêt pour les curieux.

L'année avançant le compte s'est mis à retweeter de plus en plus d'informations et à publier autre chose que les horaires d'ouverture : merci à Fred B., Jérôme B. et autres actifs du canal IRC.

Un compte Flickr a aussi été mis en place pendant l'été pour permettre la diffusion de photos sur les évènements et les projets du HAUM. Les accès pour administrer ces comptes sont disponibles sur demande.

Suite au constat concernant l'apparence désastreuse et le manque de contenu du site, Florent T. a décidé de créer un nouveau design pour le site. Après une première version (restée en ligne 4 mois), une nouvelle a vu le jour récemment.

Cette dernière version du design est accompagnée d'un changement de moteur de rendu. Le site est toujours un site statique (le seul contenu pseudo-dynamique étant généré en javascript) mais dispose maintenant de la possibilité d'écrire des articles de blog (répondant ainsi à des besoins précédemment exprimés par les membres) en plus des pages et d'étendre le langage de balisage utilisé (restructuredText). Le nouveau moteur est Pelican.

Merci aussi à tous ceux qui ont contribué au site web : Florent T., Fred B., Jérôme B, Baptiste E., Thomas L. et Mathieu G.

Au cours de la dernière assemblée générale, il avait été évoqué la possibilité de proposer des séances de micro-conférences (\textit{talks}) thématiques ou à sujets (partiellement) libres.
Ce projet a vu le jour cette année avec la mise en place d'un site ( http://talks.haum.org ) et la tenue de 2 séances de talks (une à sujet libre et l'autre sur le thème très actuel de l'Opendata).
Au cours de chacune de ces séances des intervenants extérieurs au HAUM ont pris la parole. À noter que pour la conférence sur l'Opendata, le HAUM a battu le record d'affluence pour un évènement à la Ruche Numérique de l'aveu même de l'équipe d'animation de cette dernière. Cette conférence ayant par la suite débouché sur l'organisation du hackathon préalablement cité.

Pour finir avec la partie "communication", le HAUM s'est muni de stickers (merci à Jérôme B. pour l'impression et à tous les contributeurs pour le design).

\subsubsection{Animations publiques}

Voici une liste des animations publiques auxqelles l'association a participé ou qu'elle a (co-)organisé :

\begin{itemize}
    \item formations arduino
    \item mise en place des conférences libres (2 conférences, 1 sujet "libre", 1 opendata)
    \item 24h du Code (partenariat ENSIM / CCI Ruche Numérique / Mozilla)
    \item 3D days (partenariat MakerShop / CCI Ruche Numérique)
    \item Siestes Teriaki (partenariat Petits Debrouillards / Teriaki)
    \item Hackathon Opendata (partenariat CG72 / CCI Ruche Numérique)
    \item Forum Jeunes (partenariat Francas72 / CCI Ruche Numérique)
\end{itemize}

\subsubsection{Projets et mode de fonctionnement}

Depuis l'arrivée du HAUM à Gare Sud, les efforts sont concentrés sur quelques projets communs (peu nombreux à la fois).

Cette méthode nous permet d'avancer plus vite sur les projets souhaités. Elle nous a permis d'être présents aux Siestes Tériaki et au Forum Jeune avec des projets attirants pour le public. 
    
\begin{itemize}
    \item AxiHAUM,
    \item le PianoStairs,
    \item le Pong 1D.
\end{itemize}

D'autres sont à venir.

\subsection{Objectifs}

L'objectif pour l'an prochain est de continuer à oeuvrer pour l'information du public au travers d'évènements, de séances de formation, etc.
Il est aussi nécessaire que de plus en plus de membres sachent utiliser les différents outils à disposition : fer à souder, matériels électriques, imprimantes 3D, fraiseuse, etc.

L'activité sur le compte cette année fut importante : l'organisation compte 27 dépôts et plusieurs centaines de commits. Nous souhaitons voir cette activité continuer et que, en plus du code lui même, un maximum de documentation soit mis en place pour permettre à n'importe qui de reproduire les projets.

La création d'un dépôt dédié aux tâches collectives (\texttt{https://github.com/haum/haum\_internal/issues} ) nous semble cohérente... reste à ce que les membres prennent l'habitude de l'utiliser.

Afin de garantir une communication cohérente et efficace, il est proposer de créer un "poste" de "responsable communication" au sein du bureau (non statutaire, ni règlementaire).
Celui-ci serait chargé de mettre à jour les différents comptes sur les réseaux sociaux, de vider les tickets "comm' externe" de \texttt{haum\_internal} en s'appuyant sur les autres
membres, ainsi que d'homogénéiser la communication du HAUM vers le monde extérieur. La création d'un tel poste permettrait aussi la mise en place d'un lien unique et impersonnel vers l'association pour d'éventuels partenaires.

\section{Rapport financier}

\subsection{Bilan}

\begin{figure}[h]
\centering
\begin{tabular}{l|c}
Quoi ? & Combien ? (EUR)  \\\hline
Reliquat $n-1$ & $110.71$\\
Cotisations & $230$ \\
Dons (F. Blain) & $100$ \\
Nom de domaine & $-9.59$\\
Fraiseuse & $-78.87$\\
LEDs & $-33.67$\\
Assurance & $-83.53$\\\hline\hline
TOTAL & $250.10$\\
\end{tabular}
\caption{Récapitulatif des recettes/dépenses}
\end{figure}

A ce jour, l'association compte 23 membres actifs à jour de leur cotisation. \\

Il est porté à connaissance de l'assemblée que les remboursements des projets sont des remboursements partiels. Il est donc essentiel d'augmenter la cotisation pour l'année prochaine, afin d'avoir un vrai pouvoir d'investissement.

Le HAUM voit aussi (comme dit plus haut) sa dotation en matériel augmentée avec des dons : le four à refusion d'une part
et la fraiseuse d'autre part, mais aussi les 3 serveurs et le SAN ainsi que les caisses de matériel en \textit{Free to
hack}.

\subsection{Objectifs}


\section{Motions}

\subsection{Amendements au règlement intérieur}

\subsubsection{Cotisation}

Compte-tenu du rapport financier ci-avant et de la difficulté que le HAUM a éprouvé pour financer les projets du hackerspace tout au long du dernier exercice, la présidence propose la ré-évaluation de la cotisation.

Un minimum de 20EUR par an est préconisé.


\end{document}
