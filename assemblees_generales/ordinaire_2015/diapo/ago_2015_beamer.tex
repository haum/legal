\documentclass[10pt, compress]{beamer}

% \documentclass[10pt, compress, handout]{beamer}
% \usepackage{pgfpages}
% \pgfpagesuselayout{4 on 1}[a4paper,landscape, border shrink=5mm]

	\usetheme[usetitleprogressbar]{m}

	\usepackage{url}
	\usepackage{eurosym}
	\usepackage{ulem} % to use a strikeout/strikethrough font
	\usepackage{booktabs}
	\usepackage{amsmath}
	\usepackage{nicefrac}
	\usepackage{color}
	\usepackage{wrapfig}
	\usepackage{tikz}

	\usepackage{pgfplots}

	\title{HAUM}
	\subtitle{Assemblée Générale Ordinaire}
	\date{17 Novembre 2015}

\begin{document}
\maketitle

\begin{frame}
	\frametitle{Au Menu}
	\setbeamertemplate{section in toc}[sections numbered]
	\tableofcontents
\end{frame}

\section{Présentation de l'association}

\section{Rapport Moral}

\subsection{Bilan}

\begin{frame}
	\frametitle{Faits importants}

	\begin{itemize}
		\item Augmentation du nombre d'adhérents : 23 $\rightarrow$ 30
		\item Beaucoup de projets
		\item Participation à de nombreux évènements
	\end{itemize}

	\pause
	\begin{block}{A propos du lieu}
		$29m^2$ de local devient trop juste... Il faut entasser moins ou trouver plus grand.
	\end{block}
\end{frame}

\begin{frame}
	\frametitle{Dons \& Remerciements}
	Dans l'ordre \pause(alphabétique)...

	\pause

	\begin{center}
		\pause Alice\\
		\pause Armelle\\
		\pause Florent\\
		\pause Jérôme\\
		\pause Laurent\\
		\pause Sebastien\\
		\pause Tifenn\\
		\pause Romain\\
		\pause Romuald\\
		\pause ... et tous les autres !
	\end{center}

	\pause

	Un énorme merci à \alert{Augustin} pour sa connexion et à \alert{Loïc et la Ruche Numérique} pour le local.
\end{frame}

\plain{\Huge{Merci}}


\subsubsection{Partenariats et Réseau}
\begin{frame}
	\frametitle{Partenariats \& Collaboration}

	\begin{itemize}
		\item Créalab \& OrganicOrchestra
		\item Cyclamaine
		\item Les Petits Débrouillards
		\item La ville du Mans
		\item Teriaki
		\item Ping
	\end{itemize}
\end{frame}


\subsubsection{Animations publiques}


\begin{frame}
	\frametitle{Animations Publiques}

	\begin{itemize}
		\item Formations arduino (avec la Ruche Numérique)
		\item HaumTalks
		\item 24h du Code (avec la Ruche \& l'ENSIM)
		\item Global Game Jam
		\item Festival Teriaki (avec \sout{le poulet} Teriaki)
		\item Repair Cafés (avec beaucoup de monde dont Cyclamaine, la ville, etc...)
		\item OpenBidouille Camp (avec Les Petits Débrouillards)
		\item Festival D (\textit{via} Ping)
	\end{itemize}
\end{frame}

\subsubsection{Projets}

\begin{frame}

	\frametitle{Projets}

	Beaucoup de projets dont les plus emblématiques :

	\begin{itemize}
		\item HAUMtinsel
		\item dHAUM/dHAUMidi
		\item PCBlastifieuse
		\item Meubles en carton
		\item Mise à jour du Pong/PolychrHAUM
		\item Rénovation du Pianostairs
	\end{itemize}
\end{frame}


\subsection{Objectifs}
\begin{frame}
	\frametitle{Objectifs}

	Parce qu'on peut toujours faire mieux :

	\begin{itemize}
		\item Plus de \alert{communication} autour des projets
		\item Désigner un \alert{référent} par partenaire
		\item Continuer les \alert{comptes-rendus} de séance
		\item Améliorer la \alert{gestion des projets}
	\end{itemize}

\end{frame}

\plain{Des questions/remarques ?}

\section{Rapport financier}

\subsection{Bilan}

\begin{frame}
	\frametitle{Comptes}
	\centering
	\begin{tabular}{cccc}\toprule
		Objet & Dépenses & Recettes & Soldes (positifs)\\\midrule
		Cotis. & & 590 & \\
		Services & & 5 & \\
		Dons (\euro) & & 83 & \\
		Dons & & Divers\footnote{Consommables (Makershop), outils, matières premières}  & \\
		Mise à dispo. & & Locaux & \\\midrule
		Achats & 326.94 & & \\
		Assu. & 86.79 & & \\
		Charges & 93.15 & & \\\midrule
		\textsc{\textbf{Totaux}} & 506.88 & 678 & 171.12\\\midrule
		Reliquat 2014 & & & 235.05\\
		Résultat & & & 406.17\\
		\bottomrule
	\end{tabular}
\end{frame}

\begin{frame}
	\frametitle{Adhésions \& Reports}
	\centering
	\begin{tabular}{ccccc}\toprule
		Année & Nombre d'Adhérents & Dépenses & Recettes & Résidu\\\midrule
		2013 & 13 & 34.29 & 145 & 110.71\\
		2014 & 23 & 205.66 & 330 & 124.34\\
		2015 & 32 & 506.88 & 678 & 171.12\\\bottomrule
	\end{tabular}

	\vspace{1cm}
	Evolution du nombre d'adhérents\\et de l'état financier du HAUM
\end{frame}

\subsection{Objectifs}

\begin{frame}
	\frametitle{Objectifs}


	\begin{block}{Points à améliorer}

		\begin{itemize}
			\item Les dons viennent toujours des membres : \alert{le HAUM} a une trésorerie et \alert{peut payer}
			\item Les dons venant toujours des mêmes personnes, il y a un \alert{risque d'apparition de tensions}
			\item Besoin de trouver une \alert{solution viable}...
		\end{itemize}

	\end{block}

	\pause

	\begin{block}{Nouvelles règles}

		\begin{itemize}
			\item Toutes les factures doivent être établies \alert{au nom du HAUM} et ne contenir que des éléments
				\alert{achetés pour le HAUM}. 
			\item Les dépenses de l'argent du hackerspace ne doivent être faites qu'après \alert{consultation des autres
				membres} et, surtout, \alert{vérification de la trésorerie}.
		\end{itemize}

	\end{block}
\end{frame}

\begin{frame}
	\frametitle{Solutions possibles}
	\begin{itemize}
		\item Augmentation de la cotisation
		\item Double cotisation
		\item Goodies
		\item Vente d'électronique
		\item Subventions/Sponsors par projets
		\item Prioriser le financement des projets
		\item Demander une rentabilité aux projets (\textit{cf /tmp/lab})
	\end{itemize}
\end{frame}

\plain{Des questions/remarques ?}

\section{Questions}

\begin{frame}
	\frametitle{Questions}
\begin{itemize}
	\item Que fait-on aux \alert{24h du Code 2016} ? Est-ce trop tard ?
	\item Refait-on une \alert{GlobalGameJam} ? Sera-t-elle publique ?
	\item Meubles en cartons : que fait-on du \alert{fauteuil} et des \alert{chutes de carton} ?
	\item Matthieu G. dit qu'il serait bon de demander des \alert{subventions/financements} auprès de la ville, département,... pour les projets à venir.
\end{itemize}

\end{frame}

\section{Election du bureau}

\end{document}

% vim: ts=2 sw=2
