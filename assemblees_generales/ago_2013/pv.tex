\documentclass[a4paper, 11pt]{article}
	\usepackage[utf8]{inputenc}
    \usepackage[french]{babel}
    \usepackage[T1]{fontenc}
    \usepackage[top=3cm,, right=2cm,bottom=2cm,left=2cm]{geometry}
    
    \title{Hackerspace de l'Université du Maine}
    \author{Procès-Verbal des Assemblées Générales Ordinaire et Extraordinaire}
    \date{Le Mardi 12 Novembre 2013}
    
    \newcommand\sep{\noindent\rule{\linewidth}{.5pt}}
    
    \newcommand{\vote}[5]{
    
    \smallskip
    \fbox{\begin{minipage}[l]{\textwidth}
    	\smallskip
        \begin{center}
        	\underline{\textsc{Vote}}
        \end{center}
        
        #1\\
        \textbf{Votants} #2\\
        \textbf{Pour} #3\\
        \textbf{Contre} #4\\
        \textbf{NSPP} #5
        
        \smallskip
        
    \end{minipage}}
        \medskip
    }
    
    \newcommand\question[2]{\underline{\textsc{#1 :}}
    
    \underline{#2}}
    
    
\begin{document}

\maketitle

\section*{Préambule}

Le HAUM tient à remercier Loïc Richer et toute l'équipe de la Ruche Numérique, tant pour l'aide qu'ils ont pu nous apporter que pour les invitations auquelles nous avons été conviés et enfin pour la salle où s'est déroulée cette Assemblée.

Merci aussi à Jérôme de nous avoir offert le nom de domaine pour cette année et à Charlotte pour son don en faveur de l'association.

Merci enfin à tous les membres du HAUM pour l'année passée et à ceux de LinuXMaine pour leur accueil toujours chaleureux.

\section{Effectifs}

\begin{itemize}
  \item DINSENMEYER Alice
  \item LECHAT Thomas
  \item GABORIT Mathieu
  \item BOURNEUF Lucas
  \item BREHERET Jérôme
  \item PERRIER Bertrand
  \item RICHER Loïc *
  \item VANNIER Laurent *
  \item LAPLAUD François *
  \item JOUBERT Pierre *
  \item HERISSON Stéphane
  \item CIRON Fabien *
  \item DUPONT Samuel
  \item EREAU Baptiste.
\end{itemize}

* Non-membre de l'association

\textbf{La séance est ouverte à 19:16}

La présidence présente le HAUM, les raisons de sa création et le concept de hackerspace.

La présidence actuelle décide d'accorder le droit de vote à tous les présents pour la durée de cette AG : tous ont un lien plus ou moins fort avec le HAUM et tous ont une forme de légitimité dans les décisions soumises au vote ce jour.

\section{Rapport Moral}

Pour une version complète des rapports moral et financier, consulter le dossier d'AG disponible sur le site internet du HAUM.

\textbf{Intervention de Loïc Richer à propos du Fablab} : L'ouverture du fablab se fait dans le cadre d'une politique de développement des espaces numériques (notament d'un quartier numérique). Il y a une volonté de lancer le Fablab au plus tôt (avant les réponses de l'appel à projet sur les EPN).

Beaucoup de membres s'investiront probablement dans le Fablab. Le HAUM a apporté et apporte son soutien car un fablab est un lieu bien équipé, fort de grandes ambitions et s'inscrivant dans une politique numérique sur Le Mans.\\

\vote{Rapport moral}{14}{14}{0}{0}

Le rapport moral est validé par l'assemblée.

\section{Rapport Financier}

Pour le rapport financier, voir le dossier d'Assemblée Générale.

\vote{Rapport Financier}{14}{14}{0}{0}


Le rapport financier est validé par l'assemblée.

\section{Motions}

\subsection{Règlement Intérieur : Modalité de réunion du bureau}

Il est proposé de n'obliger le bureau qu'à une réunion par année d'exercice mais de permettre à la moitié des membres de l'association de forcer le bureau à se réunir.

Sont donc soumises au vote les modifications suivantes :

\begin{itemize}
	\item Article 5, paragraphe premier : \textit{"Le bureau se réunit au moins \textbf{une} fois par annnée calendaire [...] sur la demande de la moitié de ses membres \textbf{ou de la majorité des membres actifs} [...]"} ;
    \item Article 5, paragraphe 2 : \textit{"[...] avant la date de réunion, \textbf{ainsi qu'à l'ensemble des membres via la mailing-list par exemple.}."}\\\\
\end{itemize}

\vote{Amendement au RI à propos des réunions de bureau}{14}{14}{0}{0}

L'assemblée adopte donc l'amendement.

\subsection{Statuts : Ajout d'un statut de membre partenaire}

Il est proposé la création d'un statut de membre partenaire ayant des droits différents de ceux des membres actifs. L'amendement est voté en 6 fois : une pour le concept général, cinq autres pour les droits à accorder.

L'assemblée propose de remplacer « virtuel » par « partenaire » dans le texte de l'amendement.

\vote{Création d'un statut de membre partenaire}{14}{14}{0}{0}

L'assemblée adopte donc l'amendement.

\vote{Possibilité de candidater aux élections du bureau}{14}{0}{12}{2}

L'assemblée refuse donc cette possibilité.

\vote{Possibilité de voter en assemblée générale ordinaire}{14}{12}{0}{2}

L'assemblée accorde donc cette possibilité.

\vote{Possibilité de voter en assemblée générale extraordinaire}{14}{0}{13}{1}

L'assemblée refuse donc cette possibilité.

\vote{Possibilité de participer aux convocations du bureau par les membres}{14}{9}{2}{3}

L'assemblée accorde donc cette possibilité.

\vote{Possibilité de participer aux votes concernant la fusion/dissolution de l'association}{14}{0}{14}{0}

L'assemblée refuse donc cette possibilité.
Le membre partenaire a un statut particulier : il a le droit de vote en assemblée générale ordinaire et peut participer faire réunir le bureau mais il n'a pas le droit d'être dans le bureau, de voter en assemblée générale extraordinaire et pour les votes concernant la fusion/dissolution de l'association.

\section{Questions diverses}

\question{A. Dinsenmeyer}{Quid des mini-conférences ouvertes au public sur des sujets divers ?}

Il semblerait que cela soit trop tôt, au vu du peu de projets aboutis.
il est proposé d'ouvrir un pad listant les projets (demande de talks, propositions de talks, etc...) et quand il y en aura assez, on lance le projet. Voir entre nous pour organiser des sessions internes pour commencer. Les membres semblent motivés.

\question{J. Breheret}{Quid des samedis bidouille ?}

Créé au début pour palier le manque de salle du HAUM, rien n'empèche de continuer les Samedis Bidouille en parallèle du reste.

Il est proposé de reprendre le 3ème samedi du mois. Essai en janvier, le 18 donc.

\question{J. Breheret}{Acheter un RaspberryPi ?}

La trésorerie demande qu'on lui envoie les informations complètes.

\question{L. Bourneuf}{Quid d'autres motions débiles ?}

Le président rapelle à son co-président qu'il est libre de re-convoquer une AGE s'il y tient vraiment... et de l'organiser aussi.

\section{Cotisation}

La trésorerie préconise de maintenir la cotisation à son montant actuel (10 EUR).

\vote{Maintien de la cotisation actuelle}{14}{12}{0}{2}

La cotisation est donc maintenue pour l'année à venir.

\section{Election du bureau}

Se présentent à l'élection au bureau :

\begin{itemize}
  \item BREHERET J.
  \item DINSENMEYER A.
  \item BOURNEUF L.
  \item GABORIT M.
  \item LECHAT T.
  \item EREAU B.
\end{itemize}

Le président de séance propose un vote d'approbation groupé (si ce vote n'obtient pas l'unanimité alors l'approbation de chaque candidat sera voté séparément).

\vote{Election au bureau}{14}{14}{0}{0}

\section{Réunion de bureau}

Alors que l'ensemble des membres du nouveau bureau sont présents, il est décidé de procéder dans l'immédiat aux élections internes à celui-ci. Pour les votes suivants, le collège électoral est réduit au bureau seul (soit 6 membres).

\vote{J. Bréhéret au poste de secrétaire}{6}{5}{0}{1}

\vote{B. Ereau au poste de trésorier}{6}{6}{0}{0}

T. Lechat \& M. Gaborit se présentent au poste de président.

\vote{T. Lechat au poste de président}{6}{1}{0}{5}

\vote{M. Gaborit au poste de président}{6}{4}{0}{2}

M. Gaborit est élu au poste de président.

T. Lechat \& L. Bourneuf se présentent au poste de vice-président. Il est proposé un nouveau vote groupé.

\vote{T. Lechat et L. Bourneuf au poste de vice-président}{6}{6}{0}{0}

\section{Composition du bureau}
\begin{description}
  \item[Président] M. Mathieu GABORIT
  \item[Vice-Présidents] \hfill
 	\begin{itemize}
        \item M. Thomas LECHAT
        \item M. Lucas BOURNEUF
    \end{itemize}
  \item[Trésorier] M. Baptsite EREAU
  \item[Secrétaire] M. Jérôme BREHERET
  % \item[Larbin administratif] M. Baptsite EREAU
  \item[Membres du bureau] \hfill
  	\begin{itemize}
        \item Mme Alice DINSENMEYER
    \end{itemize}
\end{description}

\textbf{La séance est levée à 20:41.}

\bigskip\bigskip

\sep

\bigskip\bigskip

Le présent procès-verbal est approuvé par le président du Hackerspace de l'Université du Maine.

\bigskip\bigskip

Président(e) :

\end{document}