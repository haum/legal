\documentclass[a4paper, 11pt]{article}
	\usepackage[utf8]{inputenc}
    \usepackage[french]{babel}
    \usepackage[T1]{fontenc}
    \usepackage[top=3cm,, right=2cm,bottom=2cm,left=2cm]{geometry}
    \usepackage{eurosym} %pour le symbole euro

    \usepackage{tabularx}
    \usepackage{soul}
	\usepackage{enumitem}
	\usepackage{pdfpages}
    \usepackage[page]{appendix}
    \renewcommand{\appendixpagename}{Annexes}

    \title{Hackerspace Au Mans}
    \author{Procès-Verbal des Assemblées Générales Ordinaire et Extraordinaire}
    \date{Le jeudi 11 février 2025}

    \newcommand\sep{\noindent\rule{\linewidth}{.5pt}}

    \newcommand{\vote}[5]{

        \bigskip
        \begin{center}
        \noindent\fbox{%
            \parbox{10cm}{%
                \smallskip
                \begin{center}\ul{\textsc{Vote}}\end{center}
                Objet : #1\\
                \begin{center}
                \textbf{Votants} #2\\
                \textbf{Pour} #3\\
                \textbf{Contre} #4\\
                \textbf{NSPP} #5
                \end{center}
                \smallskip
            }%
        }
        \end{center}
        \medskip
    }

    \newcommand\question[2]{\noindent\ul{\textit{\textsc{$\bullet$ #1}}}\\#2\\}

    %\ul{#2}}


\begin{document}

\maketitle

\section{Effectifs}

\begin{itemize}
	\item Personnes présentes : 
		\begin{itemize}
            \item Jérôme BRÉHÉRET
            \item Mathieu GABORIT
            \item Agathe MERCIER
            \item Franck DUHIL
            \item Sébastien CHOBLET
            \item Sébastien VALLÉE
            \item Joël PETIT
            \item Dag-Erling SMÖRGRAV
            \item Laurent VANNIER
            \item Qian WANG
            \item Florent TOUCHARD (Arrivée à 21h30)
		\end{itemize}
	\item Membres absents ayant donné procuration à Joël PETIT :
		\begin{itemize}
			\item Benoit PAU
		\end{itemize}
\end{itemize}

\bigskip

%%La présidence actuelle décide d'accorder le droit de vote à tous les présents 
%%pour la durée de cette AG.
\setlist{nosep,after=\vspace{\bigskipamount}}

\section{Assemblée générale ordinaire}
\textbf{La séance est ouverte à 21h25}
\subsection*{Préambule}

Le président lance un vote pour donner droit de vote à l'ensemble des personnes 
majeures présentes.

\vote{Droit de vote de toutes les personnes majeures présentes et représentées}{11}{11}{0}{0}

Le vote est accordé à toutes les personnes majeures présentes et représentées 
dans l'assemblée.

\subsection{Rapport Moral}

Le rapport moral (cf annexe) est présenté par le président.

\vote{Rapport moral}{12}{11}{0}{1}

Le rapport moral est validé par l'assemblée.

\subsection{Rapport Financier}

Le trésorier présente le rapport financier (cf annexe).

Les cotisations sur 2024 ont diminuées par rapport à 2023. On passe de 1251\euro{} en 
2023 à 940\euro{} en 2024. Cela peut s'expliquer par la perte de deux entreprises 
cotisantes sur l'année 2024. Pour ce début d'année 2025, les cotisations sont en bonne 
voie, car au moment de l'assemblée générale, plus de la moitié des cotisations de 2024 
étaient deja payées. Le trésorier note cependant le besoin de trouver de nouveaux 
adhérants pour générer de nouvelles cotisations. Le trésorier met l'accent sur nos 
présences à des évènements qui sont rétribués, afin de générer de la trésorerie, et 
ainsi financer les projets futurs et faire vivre l'association !

\vote{Rapport financier}{12}{11}{0}{1}

Le rapport financier est validé par l'assemblée.

\subsubsection{Vie de l'asso : Questions et remarques sur le rapport financier}

\subsection{Élection du bureau}

L'ensemble du bureau souhaite se représenter, à l'exception de Romuald CONTY, absent lors 
de cette assemblée, et sans représentant. Dag-Erling Smörgrav souhaite faire parti du bureau.
Aucune contre-indication ne vient de l'assemblée à ces annonces.

Se présentent donc à l'élection au bureau :

\begin{itemize}
	\item Jérôme BREHERET
	\item Mathieu GABORIT
	\item Florent TOUCHARD
	\item Sébastien VALLÉE
	\item Laurent VANNIER
	\item Dag-Erling SMÖRGRAV
\end{itemize}

Il est également convenu avec l'assemblée, toujours dans un but d'accélérer ce vote que ce 
dernier serait voté "en bloc".

La prochaine question mise au vote est donc : "Acceptez vous que la liste précédente soit 
membre du bureau de l'association ?".

\vote{Élection des membres du bureau}{12}{12}{0}{0}

\textbf{La séance est levée à 22h15}

\section{Réunion de bureau}

Alors que l'ensemble des membres du nouveau bureau est présent, il est décidé de procéder 
dans l'immédiat aux élections internes à celui-ci. Pour les votes suivants, le collège 
électoral est réduit au bureau seul (soit 6 membres, tous présents). Un tour de table 
des membres du bureau est organisé pour que chacun propose le poste qu'il souhaite occuper.

Après le tour de table, le vote concerne donc l'attribution de ces postes :
\begin{itemize}
    \item Jérôme BRÉHERET au poste de président
    \item Sébastien VALLÉE au poste de vice-président
    \item Mathieu GABORIT au poste de trésorier
    \item Dag-Erling SMÖRGRAV au poste de secrétaire
    \item Laurent VANNIER au poste de vice-secrétaire
    \item Florent TOUCHARD au poste de membre du bureau
\end{itemize}

\vote{Reconduction des rôles au sein du bureau}{6}{6}{0}{0}

Les rôles des membres du bureau sont donc validés.

\section{Composition du bureau}
\begin{description}
  \item[Président] Jérôme BRÉHÉRET
  \item[Vice-Président] Sébastien VALLÉE
  \item[Trésorier] Mathieu GABORIT
  \item[Secrétaire] Dag-Erling SMÖRGRAV
  \item[Vice-Secrétaire] Laurent VANNIER
  \item[Membre du bureau] Florent TOUCHARD
\end{description}

La composition du bureau est validé par l'assemblée.
\bigskip

\textbf{La séance est levée à 22h30.}

\bigskip\bigskip

\sep

\bigskip\bigskip

Le présent procès-verbal est approuvé par le président du HAUM.

\bigskip\bigskip

Président :



\clearpage
\begin{appendices}
\section{Rapport Moral}

\subsection{Vie de l'association}

Nous comptons 23 adhérants sur l'année 2024. C'est moins de cotisation au total 
que l'année précédente.
On compte peu de projets communs, mais plus de projets personnels et partagés.
Beaucoup de visiteurs "uniques" sont venus, par curiosité pour découvrir le lieu.
Nous notons et apprecions également les efforts de Hugues de ST Microelectronics, qui
s'est démené pour nous obtenir du matériel réformé. Au final, le matériel s'est révélé 
non fonctionnel, mais l'effort a été très apprécié du côté du bureau, qui note que cela 
à permis de maintenir la relation avec ST Microelectronics.

Un point sera à faire avec le plateau LMI, afin de mettre en place une convention
pour cadrer le partenariat en cours depuis 2017.

\subsection{Bilan}

\subsubsection{Évènements et projets terminés}

Sur l'année 2024, Le HAUM a participé à moins d'évènements et de projets qu'en 2023. 
Les évènements notables de 2024 sont présentés ci-après :

\paragraph{Teriaki 2024} Malheureusement, faute de personnes suffisantes sur le projet,
nous n'avons rien pu présenter pour cet évènement où nous avons habituellement notre 
place. Il ne faudrait pas manquer le RDV en 2025, pour garder cette place de choix dans 
les exposants de Teriaki !

\paragraph{Les 24 heures du code 2024} Le sujet proposé se nomme "L'odysée d'HAUMère" et 
consiste en 
%Décrire le projet !

\paragraph{IdiHAUM} Il a été décidé de réécrire le code d'IdiHAUM en Python, afin de 
mieux le maintenir et pouvoir ajouter de nouvelles fonctionnalités à celui-ci, comme la 
désactivation des badges d'accès ou des utilisateurs (et de tous leurs badges).

\paragraph{Les sessions bidouille} Chaque mardi soir, les portes du HAUM sont ouvertes 
à tous ceux et celles qui veulent bien les pousser.

\subsubsection{Projets et axes}

Aujourd'hui, il y a peu de projets en cours. Au niveau collectif, nous avons aujourd'hui:
\bigskip
\begin{description}
    \item[Sujet des 24h du code 2025] Création du sujet pour les 24h du code 2025, qui auront lieu les 22 et 23 mars 2025. 
    \item[Trophée des 24h du code 2025] Création des trophées pour les gagnants des 24H du code
    \item[Teriaki 2025] Une reflexion sera à entammer après les 24h du code 2025 pour concevoir un projet à présenter lors de Teriaki 2025, si nous sommes toujours convié à cet évènement. 
\end{description}

\subsection{Matériel disponible}

Le HAUM dispose actuellement du matériel suivant (entre autres):

\begin{itemize}
    \item Une découpeuse laser (Prêt de Le Mans Innovation)
    \item Une imprimante 3D (Prêt de Le Mans Innovation)
    \item Une imprimante 3D Bambulab P1S Combo
    \item Une génératrice d'étincelles (soudeuse par point)
    \item Une découpeuse vinyle Caméo (plotter de découpe)
    \item Outils pour l'électronique (dont un fer à souder acquis par le HAUM cette année)
    \item Outillage électroportatif et manuel
    \item Fraiseuse CNC 3 axes de table
    \item La majorité des projets passés
\end{itemize}

\subsection{Objectifs}


Objectif donné pour 2025 : S'organiser pour débarrasser le local pour 
désencombrer et acceuillir du nouveau matériel !


\subsection{Mot personnel du président}

%A REVOIR PAR JEROME, S'IL LE VEUT...
``Je tenais à vous remercier pour la solidarité rencontrée dans l'association,
que j'ai pu observer lors de ma convalescence, pendant laquelle j'ai toujours pu trouver
quelqu'un pour m'emmener de chez moi au local du HAUM, contribuant à ne pas me retrouver
dans un immobilisme dur à supporter. Grand merci !''

\section{Rapport Financier}
% A INCLURE !
%\includegraphics[scale=0.9]{../convocation/bilan2025_general.pdf}
%\includepdf[pages=-]{../convocation/prev2025.pdf}
\end{appendices}
\end{document}