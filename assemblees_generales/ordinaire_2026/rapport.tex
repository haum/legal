\documentclass[a4paper,11pt]{article}
\usepackage[utf8]{inputenc}
\usepackage[french]{babel}
\usepackage[T1]{fontenc}
\usepackage{url}

\title{Hackerspace Au Mans}
\author{Rapport moral 2025}

\begin{document}

\maketitle

\section{Vie de l'association}

Nous comptons 16 adhérants sur l'année 2025 dont deux entreprises.

C'est moins d'adhérents que l'année précédente, mais plus de cotisations.

\section{Bilan}

\subsection{Évènements et projets terminés}

\subsubsection{Les 24 Heures du Code 2025}

Le sujet proposé se nomme « MinHAUMtaure » et consistait à developper un programme permettant à un robot (fourni par le HAUM) à trouver et atteindre le centre d'un labyrinthe.
Le sujet complet est disponible en ligne: \url{https://github.com/haum/24hc25_robots}

Le HAUM s'est aussi chargé de réaliser les trophées pour l'événement.

\subsubsection{Siestes Teriaki 2025}

Nous avons à nouveau participé aux Siestes Teriaki avec une installation basée sur le gHaumCube.
Nous l'avions auparavant testée dans les locaux du HAUM, ce qui a beaucoup plu à Le Mans Innovation.

\subsubsection{Les sessions bidouille}

Chaque mardi soir, les portes du HAUM sont ouvertes à tous ceux et celles qui veulent bien les pousser.

\subsection{Projets en cours}

\paragraph{Résonances Arctiques}

La compagnie artistique Voix en Scène Productions nous a sollicités pour une installation basée sur le Dhaume (réimaginé en igloo illuminé) qui servira en premier lieu lors des représentations de leur spectacle « Résonances Arctiques » dans le cadre de Le Mans Sonore en janvier 2026.

\paragraph{Les 24 Heures du Code 2026}

Nous avons commencé à développer le sujet que nous comptons porter aux 24HC26.
Il est bien entendu secret.

Nous avons de nouveau été sollicités pour réaliser les trophées du concours.

\section{Locaux et matériel}

L'association est toujours hébergée par Le Mans Innovation au 57, Bd Demorieux dans le quarter Novaxud au Mans.

Nous disponsons actuellement du matériel suivant (entre autres):

\begin{itemize}
\item Une découpeuse laser (Prêt de Le Mans Innovation)
\item Une imprimante 3D Ultimaker 2 (Prêt de Le Mans Innovation)
\item Une imprimante 3D Bambu Lab P1S avec changeur de filament
\item Une génératrice d'étincelles (soudeuse par point)
\item Une découpeuse vinyle Caméo (plotter de découpe)
\item Outillage électronique (fer à souder, alimentations, instruments de mesure...)
\item Outillage électroportatif et manuel
\item Fraiseuse CNC 3 axes de table
\item Graveuse laser de table (don de la succession BRETON)
\item Machine à coudre
\item Machine à tricoter
\item La majorité des projets passés
\end{itemize}

\subsection{Objectifs}

L'objectif donné pour 2025 était de s'organiser pour débarrasser le local pour
désencombrer et acceuillir du nouveau matériel.
Puisque cela n'a pas été fait, nous reconduisons cet objectif pour 2026 !

\end{document}
