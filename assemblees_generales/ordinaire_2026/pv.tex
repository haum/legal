\documentclass[a4paper,11pt]{report}
\usepackage[utf8]{inputenc}
\usepackage[french]{babel}
\usepackage[T1]{fontenc}
\usepackage{url}
\usepackage{pdfpages}

\title{Hackerspace Au Mans}
\author{Procès-verbal de l'assemblée générale ordinaire 2026}
\date{Mardi 10 février 2026}

\begin{document}

\maketitle

\chapter*{Assemblée générale ordinaire}

\section*{Ouverture de la séance}

La séance est ouverte à 21h30 dans la salle de réunion de Le Mans
Innovation.

\section*{Effectifs}

Sont présents :

\begin{itemize}
\item Jérôme BRÉHÉRET
\item Sébastien CHOBLET
\item Franck DUHIL
\item Mathieu GABORIT
\item Benoît PAU
\item Joël PETIT
\item Dag-Erling SMÖRGRAV
\item Sébastien VALLÉE
\item Laurent VANNIER
\end{itemize}

Sont représentés :

\begin{itemize}
\item Florent TOUCHARD (membre du bureau), ayant donné procuration à
  Jérôme BRÉHÉRET
\end{itemize}

Toutes les personnes présentes ou représentées sont membres actifs à
jour de leurs cotisations.

\section*{Rapport moral}

Le président Jérôme BRÉHÉRET présente le rapport moral (voir annexe).

L'assemblée l'approuve à l'unanimité.

\section*{Rapport financier}

Le trésorier Mathieu GABORIT présente le bilan financier pour 2025 et
le budget prévisionnel pour 2026 (voir annexes).

L'assemblée l'approuve à l'unanimité.

\section*{Questions et remarques}

\subsection*{Horaires}

Une discussion a lieu au sujet des horaires des séances bidouille.  Il
est entre autre question de commencer à 19h plutôt qu'à 21h pour mieux
accomoder celleux qui peuvent difficilement s'absenter de chez eux si
tard.  Officiellement, le bureau prend acte de la discussion mais
rien ne change dans l'immédiat.  Officeusement, plusieurs membres
présents se portent volontaires pour essayer d'assurer une présence
dès 19h.

\subsection*{Gestion de la porte du local}

Le local a deux portes, dont une est équipée depuis plusieurs années
d'un lecteur de cartes RFID.  Nombre de ces cartes ont, au fil des
années, été confiées à des personnes qui ne sont plus actives et n'ont
jamais été restituées.  Le bureau suggère les mesures suivantes:

\begin{enumerate}
\item Condamnation de la serrure, de sorte qu'on ne puisse entrer
  qu'avec une carte RFID.
\item Désactivation immediate des cartes dont les détenteurs n'ont
  pas cotisé en 2025.
\item Désactivation à la date du 15 mars des cartes dont les
  détenteurs ont cotisé en 2025 mais pas en 2026.
\end{enumerate}

\section*{Élection du bureau}

L'intégralité du bureau se représente.

Le bureau est réélu à l'unanimité.

\section*{Levée de la séance}

La séance est levée à 23h20.

\chapter*{Réunion du bureau}

Il est décidé de procéder immédiatement aux élections internes au
bureau.  L'attribution des postes reste inchangée, c'est-à-dire :

\begin{figure}[h]
\begin{tabular}{ll}
  Jérôme BRÉHÉRET & Président \\
  Sébastien VALLÉE & Vice-président \\
  Mathieu GABORIT & Trésorier \\
  Dag-Erling SMÖRGRAV & Secrétaire \\
  Laurent VANNIER & Vice-secrétaire \\
  Florent TOUCHARD & Membre du bureau \\
\end{tabular}
\end{figure}

\appendix

\chapter{Rapport moral}

\section*{Vie de l'association}

Nous comptons 16 adhérents sur l'année 2025 dont deux entreprises.

C'est moins d'adhérents que l'année précédente, mais plus de cotisations.

\section*{Bilan}

\subsection*{Évènements et projets terminés}

\subsubsection*{Les 24 Heures du Code 2025}

Le sujet proposé se nomme « MinHAUMtaure » et consistait à developper un programme permettant à un robot (fourni par le HAUM) à trouver et atteindre le centre d'un labyrinthe.
Le sujet complet est disponible en ligne: \url{https://github.com/haum/24hc25_robots}

Le HAUM s'est aussi chargé de réaliser les trophées pour l'événement.

\subsubsection*{Siestes Teriaki 2025}

Nous avons à nouveau participé aux Siestes Teriaki avec une installation basée sur le gHaumCube.
Nous l'avions auparavant testée dans les locaux du HAUM, ce qui a beaucoup plu à Le Mans Innovation.

\subsubsection*{Les sessions bidouille}

Chaque mardi soir, les portes du HAUM sont ouvertes à tous ceux et celles qui veulent bien les pousser.

\subsection*{Projets en cours}

\paragraph{Résonances Arctiques}

La compagnie artistique Voix en Scène Productions nous a sollicités pour une installation basée sur le Dhaume (réimaginé en igloo illuminé) qui servira en premier lieu lors des représentations de leur spectacle « Résonances Arctiques » dans le cadre de Le Mans Sonore en janvier 2026.

\paragraph{Les 24 Heures du Code 2026}

Nous avons commencé à développer le sujet que nous comptons porter aux 24HC26.
Il est bien entendu secret.

Nous avons de nouveau été sollicités pour réaliser les trophées du concours.

\section*{Locaux et matériel}

L'association est toujours hébergée par Le Mans Innovation au 57, Bd Demorieux dans le quarter Novaxud au Mans.

Nous disponsons actuellement du matériel suivant (entre autres):

\begin{itemize}
\item Une découpeuse laser (Prêt de Le Mans Innovation)
\item Une imprimante 3D Ultimaker 2 (Prêt de Le Mans Innovation)
\item Une imprimante 3D Bambu Lab P1S avec changeur de filament
\item Une génératrice d'étincelles (soudeuse par point)
\item Une découpeuse vinyle Caméo (plotter de découpe)
\item Outillage électronique (fer à souder, alimentations, instruments de mesure...)
\item Outillage électroportatif et manuel
\item Fraiseuse CNC 3 axes de table
\item Graveuse laser de table (don de la succession BRETON)
\item Machine à coudre
\item Machine à tricoter
\item La majorité des projets passés
\end{itemize}

\subsection*{Objectifs}

L'objectif donné pour 2025 était de s'organiser pour débarrasser le local pour
désencombrer et acceuillir du nouveau matériel.
Puisque cela n'a pas été fait, nous reconduisons cet objectif pour 2026 !

\chapter{Bilan financier 2025}

\includepdf{bilan_financier_2025.pdf}

\chapter{Budget prévisionnel 2026}

\includepdf{budget_previsionnel_2026.pdf}

\end{document}
