\documentclass[a4paper, 11pt]{article}

    % Document largement inspiré des statuts du LOL :
    %     http://labolyon.fr
    %
    % Merci à eux !
    %
    % Versions :
    % [0.1] HAUM
    %     sam. oct. 13 12:22:54 CEST 2012

    \usepackage[utf8]{inputenc}
    \usepackage[T1]{fontenc}
    \usepackage{geometry}
    \geometry{top=3cm, bottom=2cm, left=2cm, right=2cm}

    \title{Statuts}
    \author{}
    \date{}

    \renewcommand{\thesection}{Article \arabic{section}~~~-}
    \newcommand{\nomHS}{HAUM}
    \newcommand{\dateAGC}{mardi 13 novembre 2012}
    \newcommand\sep{\noindent\rule{\linewidth}{.5pt}}

\begin{document}

    \maketitle

\section{Dénomination} % {{{

Il est fondé entre les adhérents aux présents statuts une association régie par la loi du 1901 et le décret du 16 août
1901, ayant pour titre «\nomHS».

% }}}

\section{Objet} % {{{

Cette association a pour buts de :

\begin{itemize}
    \item promouvoir la philosophie du Libre et du DIY\footnote{Do It Yourself : réaliser des choses par
        soi-même} ainsi que l'innovation par la réalisation de projets personnels et/ou communs dans un cadre
        communautaire.
    \item favoriser la transmission non-marchande de connaissances et savoir-faire, notamment en publiant le travail des
        adhérents lié à des projets réalisés ou en cours.
    \item informer sur les domaines explorés au cours de projets via des partenariats, des présentations ou des ateliers
        (workshops).
\end{itemize}

% }}}

\section{Siège social} % {{{

Le siège social est fixé à Le Mans.
% il faut une adresse précise
% il ne me semble pas

% }}}

\section{Durée} % {{{

L'association est créée pour une durée illimitée.

% }}}

\section{Membres} % {{{

Est considéré membre actif de l'association toute personne physique qui :

\begin{itemize}
    \item adhère aux buts de l'association
    \item accepte ses statuts et son règlement intérieur
%    **********************************************************
    \item voit sa candidature acceptée par le bureau
    \item est à jour de cotisation
%    **********************************************************
%    j'enleverais ces deux éléments des status pour les mettre dans le réglement inter
% Ben, oui et non, ils pourraient aller dans le RI, mais l'avantage,
% c'est que les statuts regroupent comme ça toutes les modalités d'adhésion
\end{itemize}
% **************************************************************
Les modalités de demande d'adhésion sont précisées dans le règlement intérieur.
En cas de refus d'une demande d'adhésion, le bureau doit motiver sa décision.
% ****************************************************************
% idem pour ca !
% les statuts étant plus difficiles à modifier que le RI, il vaut mieux que ce genre de "sécurités" soient dans les statuts plus que le règlement

Est considéré membre virtuel de l'association toute organisation ou personne morale adhérant selon des modalités
identiques à celles des membres actifs (précisées ci-avant).

Les membres partenaires peuvent :
\begin{enumerate}
    \item voter en assemblée générale ordinaire
    \item participer aux convocations du bureau par les membres
\end{enumerate}

Tous les autres droits dévolus aux membres actifs leur sont refusés.
% }}}

\section{Cotisation} % {{{

Le montant de la cotisation est fixé chaque année lors de l'assemblée générale ordinaire.
La périodicité et le mode de paiement de la cotisation sont précisés dans le règlement intérieur.

% }}}

\section{Démission, radiation} % {{{

La qualité de membre de l'association se perd par :

\begin{itemize}
    \item démission communiquée au bureau;
    \item radiation pour motif grave;
    \item décès
\end{itemize}

La radiation est prononcée par le bureau selon les modalités précisées par le règlement intérieur.

% }}}

\section{Bureau} % {{{

L'association est dirigée par un bureau, constitué au minimum de trois membres issus des membres actifs.
Les membres du bureau sont élus individuellements par l'assemblée générale ordinaire pour une durée d'un an et sont
rééligibles plusieurs fois.

À chaque changement de sa composition, et au moins une fois par an, le bureau élit parmi ses membres un président, un
trésorier et un secrétaire.
Si le bureau le juge nécessaire, il élira un ou plusieurs vice-présidents, un ou plusieurs vice-trésoriers et un ou
plusieurs vice-secrétaires.


En cas de défaillance d'un ou plusieurs membres du bureau (décès, démission ou radiation), réduisant l'effectif du
bureau à moins de trois personnes, le bureau organisera une assemblée générale extraordinaire pour procéder à l'élection
du ou des remplaçants.

Les fonctions ne sont pas cumulables.

% }}}

\section{Assemblée Générale Ordinaire} % {{{

L'assemblée générale ordinaire comprend les membres actifs de l'association.

Elle se réunit au moins une fois par année calendaire, à une date fixée par le bureau.

Quinze jour avant la date fixée, les membres actifs sont convoqués par soit :

\begin{itemize}
    \item un e-mail publié sur la mailing-list officielle de l'association
    \item un courrier envoyé à leur domicile s'ils en font la demande
\end{itemize}

L'ordre du jour doit apparaitre sur la convocation.

Les membres actifs dans l'impossibilité de se rendre à l'assemblée générale peuvent donner procuration à un autre membre
actif de l'association pour les représenter selon les modalités précisées dans le règlement intérieur.

L'assemblée délibère sur tous les points inscrits à l'ordre du jour. Les décisions sont prises selon les modalitées
précisées dans le règlement intérieur.

% }}}

\section{Assemblée Générale Extraordinaire} % {{{

L'assemblée générale extraordinaire comprend les membres actifs de l'association.

Elle peut être réunie par simple demande du président, de la majorité des membres du bureau ou de la majorité des
membres actifs.

La convocation, la délibération et le vote se font suivant les mêmes modalités que pour l'assemblée générale ordinaire.

Seule l'assemblée générale extraordinaire est compétente pour modifier les statuts, décider la dissolution, la fusion de
l'association avec toute autre association poursuivant le même objet ou les mêmes orientations.

% }}}

\section{Recherche de consensus} % {{{

Le processus de décision le plus courant au sein de l'association doit être le consensus. Avant qu'une décision du
bureau ou d'une assemblée générale ne soit soumise au vote majoritaire, un effort raisonable de recherche de consensus
doit avoir été effectué.

% }}}

\section{Règlement intérieur} % {{{

Le fonctionnement de l'association est régi par son règlement intérieur adopté par l'assemblée constitutive.

Les mofications ultérieures du règlement intérieur se font suivant les modalitées précisées dans la version en vigueur
de ce même règlement.

% }}}

\section{Dissolution} % {{{

En cas de dissolution volontaire ou forcée, l'assemblée générale extraordinaire statue de la dévolution du patrimoine de
l'association.

Elle désigne les organismes à but non-lucratif poursuivant des objectifs analogues qui recevront le reliquat de l'actif
après paiement de toutes dettes et charges de l'association et de tous frais de liquidation. Elle nomme pour assurer les
opérations de liquidation, un ou plusieurs liquidateurs parmi les membres actifs qui seront investis de tous les
pouvoirs nécessaires.

% }}}


\bigskip\bigskip

\sep

\bigskip\bigskip

Les présents statuts ont été approuvés par l'assemblée générale constitutive du \dateAGC.

\bigskip\bigskip

Président(e) :


\bigskip\bigskip

Trésorier(ère) :

\end{document}

