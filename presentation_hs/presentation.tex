\documentclass[a4paper, 11pt]{report}

    \usepackage[utf8]{inputenc}
    \usepackage[T1]{fontenc}
    \usepackage[french]{babel}

    % header/footer
    \usepackage{fancyhdr}
    \lhead{}
    \chead{\bfseries Un Hackerspace au Mans}
    \rhead{}
    \lfoot{Octobre 2012}
    \cfoot{}
    \rfoot{\thepage}
    \renewcommand{\headrulewidth}{0.4pt}
    \renewcommand{\footrulewidth}{0.4pt}

    \newcommand{\hs}{{\itshape hackerspace}}
    \newcommand{\hsss}{{\itshape hackerspaces}}
    \newcommand{\hw}{{\itshape hardware}}
    \newcommand{\sw}{{\itshape software}}

    \usepackage{url}


    % Premiere page
    \title{Un Hackerspace au Mans}
    \author{}
    \date{Octobre 2012}

\begin{document}

    \maketitle

    \tableofcontents
    \newpage

    \section{Un Hackerspace ?}

    \subsection{Présentation}

La définition d'un \hs{} est assez difficile à donner. Un {\it hacker} étant littéralement un {\it bidouilleur}, je
pense qu'un \hs{} pourrait se définir comme un {\it lieu à bidouilleurs}.

En fait, un \hs, c'est un endroit où des gens se retrouvent pour bricoler, coder, résoudre des problèmes, réfléchir,
créer et partager un bon moment ensemble.

Il est difficile de donner une définition au terme \hs{} parce que celle-ci varie en fonction des personnes
présentes.

La définition de Wikipedia\footnote{\url{http://fr.wikipedia.org/wiki/Hackerspace}} me semble plutôt bonne :

\begin{quotation}
Un hackerspace, hacklab ou media hacklab est un lieu où des gens avec un intérêt commun (souvent autour de
l'informatique, de la technologie, des sciences, des arts\ldots) peuvent se rencontrer et collaborer. Les Hackerspaces
peuvent être vus comme des laboratoires communautaires ouverts où des gens (les hackers) peuvent partager ressources
et savoir.
\end{quotation}

Des hackerspaces existent un peu partout dans le monde (on en dénombre plus d'un millier) et notament en France.
Ici, on compte environ une vingtaines de \hsss{} officiels et actifs et une myriades d'associations y ressemblant.
Enfin, un certain nombre de \hsss{} sont en projet ou existent autour du Mans :

\begin{itemize}
    \item META à Blois\footnote{\url{http://www.facebook.com/meta.blois}}
    \item Breizh Entropy Lab à Rennes\footnote{\url{http://breizh-entropy.org/}}
    \item PiNG à Nantes\footnote{\url{http://www.pingbase.net}}
    \item Labomedia à Orléans\footnote{\url{http://labomedia.org/}}
    \item Ipéfix à Angers
\end{itemize}

    \subsection{Hackerspace ou Fab Lab ?}

    La différence principale entre un \hs{} et un {\itshape Fab Lab} est que le second est dédié à la fabrication d'objets.

Le projet {\itshape Fab Lab} a été lancé par le MIT autour de cette idée, et la création d'un tel lab est sujette à plusieurs
soucis :

\begin{itemize}
    \item besoin d'outils
    \item besoin de machines particulières
    \item besoin de compétences sur ces machines
\end{itemize}

Le \hs{} a l'avantage d'être plus général, mais aussi d'être beaucoup moins contraint par la présence ou non de tels objets en
(presque) libre service.

    \section{Notre projet}

Après une année d'existence plus ou moins informelle comme sous-groupe de l'association Asimov à l'IUP MIME, nous
aimerions faire grandir un peu le \hs{} et notament l'ouvrir aux personnes extérieures.

Voilà, en quelques points, l'aspect du projet.

    \subsection{Du matériel, du code, des hackers}

C'est la base d'un bon nombre de \hsss{} à travers le monde.

Les \hsss{} ont une forte {\it "tradition"} \hw, et, alors que certains ont déjà utilisé des plateformes comme
l'Arduino\footnote{\url{http://arduino.cc}}, nous aimerions continuer à réfléchir autour de systèmes électroniques.

De ce côté, il s'agit de s'approprier l'électronique pour le simple plaisir de la découverte ou bien au sein de projets
plus importants.

\medskip

Le \hw{} est {\bf le} lien entre un système et le monde réel et c'est une composante primordiale dans les activités d'un
\hs.

Le code vient ensuite quand il s'agit de bidouiller sur ordinateur ou de faire prendre vie à nos systèmes : on parle
alors de \sw.

Bien sûr, il peut s'agir de création de programmes complets, de protoypes, mais aussi de modification sur des projets
{\itshape open-source} ou de contribution à ceux-ci.

L'idée est d'utiliser ce que le logiciel libre nous propose et surtout d'y contribuer.

\medskip

Quand le titre parle de {\it "hackers"}, il s'agit bien évidement des membres du \hs.
Des personnes cherchant des solutions à des problèmes, voulant créer ou adapter quelque chose et travaillant ensemble
pour ça.

Un \hs{} n'est pas un lieu inerte et ce sont ses membres qui le font vivre. On retrouve généralement des valeurs fortes
d'entraide et de travail en équipe.

Pour des étudiants, un \hs, c'est le moyen d'appliquer et d'approfondir ce qu'ils apprennent.
Pour les autres, c'est un vecteur d'apprentissage et d'expérimentation unique.

    \subsection{Découverte}

    Il est courant qu'un \hs{} organise des évènements en plus de ses "séances". Ça peut être des ateliers ({\it
workshops}) pour ceux qui veulent apprendre, mais aussi des initiations à certains concepts ou logiciels.

Un \hs, c'est un concentré de connaissances sur différents sujets dont certains reviennent particulièrement :

\begin{itemize}
    \item les logiciels libres
    \item GNU/Linux
    \item certaines astuces DIY\footnote{Do It Yourself : une maxime conseillant aux gens d'essayer de faire les choses
    eux-mêmes}
    \item la vie privée dans la technologie
    \item etc\ldots
\end{itemize}

L'avant dernier point peut étonner, mais oui, un \hs{} peut aussi permettre aux gens d'apprendre à mieux gérer leur vie
privée dans notre société technologique moderne. Cela peut prendre tout un tas de formes, mais par exemple, on peut
discuter de PGP/GPG\footnote{système de signature/chiffrement par clé publique/privée. Notament utile pour les mails},
de certificats et de communication chiffrée\ldots

    \subsection{Expérimentation}

Un \hs{} permet enfin à ses membres de tester leur idées, d'en discuter. C'est un lieu où l'on peut expérimenter, que l'on
parle de \sw, de \hw{} ou bien d'idées plus générales, ne traitant pas de technologie. Il n'est pas rares de voir des
\hsss{} publier des documents traitant d'autre chose que de technologie.

C'est là toute la force d'un \hs{} : en ne se différenciant pas, il permet à ses membres d'explorer un maximum de voies et
ainsi d'apprendre dans un maximum de sujets.

    \subsection{Et pour le reste ?}

Il y aurait des dixaines de choses à dire sur la notion de \hs{} et ses implications, mais ce n'est pas l'objet (ce serait
beaucoup trop long).

Je voudrais simplement discuter d'un point plutôt important : comment le \hs{} se lie-t-il au reste du monde ?

L'image typique évoquée par le mot "\hs" représente une {\it "bande de barbu(e)s dans un garage"} seulement, non, le
\hs{} a aussi une forme d'implication sociale.

Tout d'abord, le sujet de l'OpenData est souvent discuté dans les \hsss, ceux-ci permettent à la population de
s'approprier plus facilement lesdites données. Si l'on se place du côté des acteurs publiant les données, un \hs,
c'est un vecteur de commentaires intéressant. Récupérer des données, les traiter et les redistribuer, tout cela fait
partie du mouvement {\it Datalove} dans lequel nombre de \hsss{} se reconnaissent.

Le deuxième "pont" vers le monde réel apparait sous forme de présentations. Il est intéressant pour un \hs{} d'organiser
des ateliers ou micro-conférences ouverts au public de façon régulière afin d'avoir une existence propre dans son
environnement.

Cela permet au \hs{} de se faire connaître et aux gens d'apprendre des choses (soit par la pratique soit au cours d'une
présentation orale).

    \section{Valeurs}

Il me semble important, pour finir ce document de dresser une liste des valeurs propres aux \hsss{} :

\begin{itemize}
    \item curiosité/ouverture d'esprit
    \item liberté individuelle
    \item partage
    \item travail/effort
    \item rigueur
    \item persévérence
    \item réflexion
\end{itemize}

Je vais conclure ainsi cette (courte) présentation du concept de \hs{} et de ce que nous aimerions en faire.

Si vous souhaitez plus de précisions ou simplement en discuter, vous pouvez :

\begin{itemize}
    \item utiliser la mailing liste provisoire\footnote{\url{https://groups.google.com/forum/#!forum/hs_lemans}}
    \item rejoindre le canal IRC pour dicuter autour du projet :
        
        \#hs\_lemans @ irc.freenode.net\footnote{\url{http://webchat.freenode.net/}}
    \item me contacter directement par mail à l'adresse \url{mathieu@matael.org}
\end{itemize}

\end{document}
